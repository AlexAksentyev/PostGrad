\documentclass[14pt]{extarticle}

\usepackage{mystyle}
\usepackage{PolDefs}
\usepackage{subfiles}

\newcommand{\SMatrix}{\mathcal{S}}
\newcommand{\TMatrix}{\mathcal{T}}
\newcommand{\IMatrix}{\mathcal{I}}

\newcommand{\ampl}[2]{\langle #1 \vert #2 \rangle}

\newcommand{\SrcNm}[1]{\textsc{#1}}

\begin{document}


\begin{titlepage}
	\begin{center}
		
		{Федеральное государственное автономное образовательное учреждение\\			
			высшего профессионального образования\\		
			«Национальный исследовательский ядерный университет «МИФИ»} \\[2.4cm]
		
		
		
		% Title
		\textsc{\textbf{История исследования фундаментальных симметрий} \\[2.4cm] }
		
		
	\end{center}
	
	
	\begin{flushright}
		Реферат по истории науки аспиранта кафедры №14
		\begin{tabular}{rr}
			Выполнил: & Александр Евгеньевич Аксентьев \\
			&	специальность 01.04.20 \\					
			Научный руководитель: & Доц., к.ф.-м.н.  С.М. Полозов 	\\
			Преподаватель группы: & Доц, к.филос.н. Н.Б. Миронова
		\end{tabular}
		
	\end{flushright}
	
	\vfill
	
	
	\begin{center}
		\the\year{}
	\end{center}
	
	
	
\end{titlepage}

\tableofcontents
\pagebreak

\section*{Зачем изучать симметрии}

С обще-философской точки зрения, проверка симметрий полезна в связи с тем, что симметрии выполняют следующие важные роли в науке:
\begin{enumerate}
	\item Классификация. Наилучшим примером такого использования симметрий служит классификация элементарных частиц в терминах групп симметрии фундаментальных физических симметрий. 
	\item Ограничение физических теорий. Инвариантность относительно преобразования накладывает жёсткие ограничения как на величины, которые могут входить в теорию, так и на форму её фундаментальных уравнений.
	\item Унификация. Математический аппарат теории групп, используемый для описания симметрий, открывает возможность для унификации различных симметрий, путём унификации соответствующих групп преобразований; таким образом теории, основанные на симметриях, так же могут быть унифицированы.
\end{enumerate}

Ограничительная роль симметрий проявляется, например, в несоответствии предсказаний Теории Большого Взрыва (физические законы которой симметричны относительно фундаментальных симметрий квантовой физики), и Барионной Асимметрией Вселенной: колоссальным превалированием материи над антиматерией в наблюдаемой вселенной.

В 1967 году, советский физик Андрей Дмитриевич Сахаров предложил три условия, которым должно удовлетворять барион-генерирующее взаимодействие, чтобы в его результате материя и анти-материя производились в разном соотношении. Одним из этих условий является нарушение физическими законами C- и CP-симметрий.

\section{Симметрия}
\subsection{Что такое симметрия}

Слово симметрия --- производное от слов \emph{сим}, совместно, и \emph{метро}, измерять, --- изначально обозначало отношение соизмеримости. Вскоре оно приобрело более общий смысл отношения \emph{пропорциональности}, смысл которого был в объединении различных \emph{элементов} в \emph{единую систему}. Таким образом, с самого зарождения, концепция симметрии заняла центральную роль в естественных теориях.~\cite{SEP.Symmetry}

Современная интерпретация понятия симметрии даётся как инвариантность относительно заданного преобразования. Это значение понятие симметрии приобрело в 17 веке, и оно было основано не на \emph{пропорциональности}, а на \emph{равенстве} соотносимых элементов (например, равенстве левой и правой частей фигуры).\footnote{Таким образом, отношение симметрии было обобщено ещё раз, путём разделения на две части: конкретное отношение пропорциональности было заменено равенством, с добавлением арбитрарного преобразования, относительно которого имеет смысл соотношение.} Ключевым свойством этой интерпретации является \emph{заменимость} симметричных частей \emph{относительно общего целого}.

Следующим шагом в эволюции концепции симметрии стало развитие алгебраического понятия \emph{группы}; оказалось, что преобразования симметрии формируют группы,\footnote{Группой называется множество $G$ вместе с операцией $\star$ на нём, удовлетворяющей аксиомам группы.} то есть 
\begin{inparaenum}[(1)]
	\item композиция преобразований симметрии есть преобразование симметрии,
	\item она ассоциативна,
	\item преобразование идентичности не нарушает симметрию, и
	\item любое преобразование симметрии обратимо.
\end{inparaenum}

Определение симметрии как инвариантности относительно определённой группы преобразований сделало это понятие применимым не только в отношении геометрических фигур, но также и математических \emph{выражений}, в частности --- уравнений динамики.~\cite{SEP.Symmetry}

\subsection{Роль симметрии в физических теориях}

Первым явным применением принципа инвариантности физических уравнений стала процедура решения уравнений динамики, разработанная немецким математиком Карлом Густавом Якоби. Подход Якоби заключался в использовании канонических преобразований переменных, не изменяющих уравнения Гамильтона системы. Таким образом, изначальная проблема сводилась к эквивалентной (относительно Гамильтоновой формулировки), но более простой. Этот подход привёл к исследованию физических теорий на предмет их преобразуемости; примерами таких исследований служат:
\begin{inparaenum}[(i)]
	\item исследования инвариантов канонических преобразований (скобки Пуассона, интегралы Пуанкаре), а также
	\item теория непрерывных канонических преобразований норвежского математика Софуса Ли.
\end{inparaenum}

Альтернативным подходом обозначенному выше, при котором изучаются симметрии \emph{заданных} физических законов, является \emph{постулирование} определённой симметрии, с последующим поиском законов, удовлетворяющих этой симметрии. Поворотным моментом в методологии науки стала публикация Альбертом Эйнштейном Специальной Теории Относительности, построенной на основании постулата об универсальности непрерывных симметрий пространства-времени.~\cite{SEP.Symmetry}

\subsection{Симметрии в квантовой механике}

Наиболее успешным образом принципы симметрии применяются в квантовой физике. Причиной тому послужили два факта: 
\begin{inparaenum}[(1)]
	\item использование математического аппарата теории групп для описания симметрии, а также
	\item использование векторного пространства в описании квантовой физики.
\end{inparaenum}

Являясь алгебраической структурой, группа может быть представлена набором матриц на векторном пространстве;\footnote{В теории представлений, представлением группы $G$ называется функция из $G$ во множество преобразований векторного пространства. Таким образом, \emph{каждому} элементу группы сопоставляется матрица преобразования векторного пространства.} а поскольку пространство состояний системы в квантовой механике есть векторное пространство, любое состояние сложной системы может быть сведено к линейной комбинации состояний простых систем, преобразуемых матрицей операции симметрии определённым образом.

Для того, чтобы исследовать фундаментальные симметрии, в ядерной физике проводятся эксперименты по рассеянию ускоренного пучка элементарных частиц. При проведении экспериментов по рассеянию, состояния частиц (описываемые амплитудами вероятности) связаны между собой через матрицу рассеяния~\cite{Symmetries}
\[
\SMatrix_{fi} \equiv \ampl{f_{(out)}}{i_{(in)}},
\]
где $\ket{i_{(in)}}$ обозначает $i$-ое входное состояние, $\ket{f_{(out)}}$ -- $f$-ое выходное.

Матрица рассеяния также выражается (линейно) через матрицу перехода 
\[
\TMatrix_{fi} = \bra{f}H_{int}\ket{i_{(in)}} = \bra{f_{(out)}}H_{int}\ket{i},
\]
где $H_{int}$ обозначает Гамильтониан взаимодействия.

Поэтому, формально, о симметриях квантовой теории можно говорить как о симметриях матрицы перехода (или рассеяния) под воздействием соответствующего унитарного оператора:
\[
\Mod{\bra{f'}\TMatrix\ket{i'}} = \Mod{\bra{f}U^\dagger\TMatrix U\ket{i}} = \Mod{\bra{f}\TMatrix\ket{i}}.
\]


\subsubsection{C-симметрия}

Сопряжением знака называется преобразование, ассоциированное с заменой частиц на анти-частицы, т.е. при котором все заряды меняют знак, в то время как другие величины остаются неизменны. Преобразование сопряжения знака формально описывается оператором $C$, действие которого на состояние с определённым значением импульса $\vec{p}$, проекции спина $s$ и зарядом $q$ есть
\[
C\ket{\vec{p}, s, q} = \eta_C\ket{\vec{p}, s, -q}, \quad \Mod{\eta_C}^2 = 1.
\]

В терминах матрицы рассеяния, C-симметрия выражается как $\sqB{C,\SMatrix}=0$, следствием чего является (если обозначать анти-частицы через надчёркивание) 
\[
\SMatrix_{fi} = \eta_C(f)^*\eta_C(i)\SMatrix_{\bar{f}\bar{i}}.
\]

Из симметрии сопряжения знака обязательно следует наличие античастицы, для любой нейтрильной частицы, которая подчиняется тем же самым законам что и частица; с той лишь разницей, что знаки всех её внутренних зарядов заменены на противоположные. Тем не менее, наличие античастиц не достаточно для проверки нарушения $C$-симметрии, что связано с более фундаментальной $CPT$-симметрией; тесты $C$-симметрии связаны со свойствами взаимодействий частиц.~\cite[стр. 98]{Symmetries}

\subsubsection{P-симметрия}

Преобразование чётности определяется как отражение всех пространственных координат относительно точки отсчёта. В квантовой теории, этому преобразованию соответствует (унитарный) оператор $P$, такой что
\[
P\ket{\vec{p}, s, q} = \eta_P\ket{-\vec{p}, s, q}, \quad \Mod{\eta_P}^2 = 1.
\]

Для взаимодействий не нарушающих чётность, т.е. таких что $\sqB{P, \SMatrix} = 0$, на матрицы рассеяния до преобразования $\SMatrix_{fi}$ и после $\SMatrix_{f_P i_P}$ есть
\[
\SMatrix_{fi} = \eta_P(f)^*\eta_P(i)\SMatrix_{f_P i_P}.
\]

В связи с тем что оператор спина $S$ преобразуется как $P^\dagger S P = S$, $P$-симметрия теории подразумевает отсутствие $P$-нечётных слагаемых (таких как $\dotprod{\vec{S}}{\vec{p}}$) в сечениях взаимодействий процесса.~\cite[стр. 30]{Symmetries}\\

\begin{Remark}[Философское значение нарушения P-симметрии]
Нарушение P-симметрии открывает новую главу в споре между субстанциальной и реляционной концепциями пространства. 

Субстанциальная позиция, в пользу которой аргументировал Исаак Ньютон, состоит в том, что концепция пространства указывает на \emph{объект}, существующий независимо от наличия материи во вселенной; Готтфрид Лейбниц, с другой стороны, утверждал что то, что мы называем пространством --- есть продукт нашего разума: набор \emph{отношений} между объектами, не имеющий смысла в их отсутствии.~\cite[разд. 6.1]{ConceptOfSpace} 

Для реляциониста, материя --- источник пространства. Каким образом, тогда, материя может выбирать между различными направлениями, если направления не существуют (сами по себе)? Каким образом частица `знает' что она должна лететь \emph{влево}, если \emph{лево} не существует пока она не полетит?\footnote{Автор этой работы не понимает в чём проблема. Электрон может и не знает лево-право, но он знает куда смотрит \emph{спин}.}
\end{Remark}

\subsubsection{T-симметрия}

Под операцией отражения времени понимается отражение \enquote{направления движения}, т.е. смена знака импульса частицы (при сохранении координаты). В квантовой физике, отражению времени соотвествует (анти-унитарный) оператор $T$, определяемый как
\[
T\ket{\vec{p}, s, q} = \eta_T\ket{-\vec{p}, -s, q},\quad \Mod{\eta_T}^2 = 1.
\] 

$T$-симметрия ($\sqB{T, \SMatrix} = 0$) обеспечивает 
\[
\SMatrix_{fi} = \eta_T(f)\eta_T(i)^*\SMatrix_{f_T i_T}.
\]

Для процесса $i\to f$, $T$-симметрия записывается как $\TMatrix_{f_T i_T} = \TMatrix_{if}$. Матрица перехода удовлетворяет уравнению 
\[
\Mod{\TMatrix_{if}}^2 = \Mod{\TMatrix_{fi}}^2 - 2\Im{\IMatrix_{fi}\TMatrix_{fi}} - \Mod{\IMatrix_{fi}}^2,
\]
где $\IMatrix$ описывает состояния между $i$ и $f$; и следовательно
\[
\Mod{\TMatrix_{f_T i_T}}^2 - \Mod{\TMatrix_{fi}}^2 = \Mod{\TMatrix_{f_T i_T}}^2 - \Mod{\TMatrix_{if}}^2 - 2\Im{\IMatrix_{fi}\TMatrix_{fi}} - \Mod{\IMatrix_{fi}}^2.
\]
В зависимости от того есть/нет взаимодействия частиц в конечном состоянии, $\IMatrix_{fi}$ не-/равно нулю, и $T$-нечётные наблюдаемые не-/имеют нулевое математическое ожидание. В связи с этим, наличие $T$-нечётных слагаемых (таких как $\dotprod{\vec S}{\rndB{\xprod{\vec p_1}{\vec p_2}}}$) в Гамильтониане системы не может само по себе свидетельствовать о нарушении $T$-симметрии.~\cite[стр. 146]{Symmetries}

\subsection{CPT-теорема} 

$CPT$-преобразование появляется при рассмотрении обращения 4-вектора, которое можно произвести с помощью преобразования Лоренца (являющегося симметрией, на сколько известно на данный момент). Интуитивно, $PT$-преобразование также должно быть эквивалентно обращению пространства-времени, однако это не так; таковым является \emph{произведение} преобразований $C, P$ и $T$ (в любом порядке) -- $CPT$, являющееся, к тому же, симметрией, по теореме CPT. 

CPT-теорема требует того, чтобы при нарушении хотя бы одной из $C$, $P$ или $T$ симметрий, по крайней мере ещё одна так же должна нарушиться. 


\section{Эксперименты по проверке симметрий}

Наиболее прямой способ проверки дискретной симметрии (коими являются $C$, $P$, $T$) -- проведение двух экспериментов, инвертированных друг относительно друга с точки зрения исследуемого преобразования.
%
Свидетельством нарушения симметрии может быть наблюдения ненулевой величины среднего значения \emph{нечётной} наблюдаемой, при соответствующем преобразовании.~\cite[стр. 12]{Symmetries}

В связи с фундаментальной важностью данного понятия в современной науке, эксперименты по проверке нарушения симметрий производятся по всему миру, со всё более совершенной аппаратурой и методами измерения. Некоторые из этих экспериментов рассматриваются далее.

\subsection{P-инвариантность}

Идея тестирования $P$-симметрии появилась в связи с невозможностью объяснения распадов в некоторых системах $K$-мезонов, в предположении о верности $P$-симметрии; так называемый $\tau-\theta$ парадокс.~\cite[стр. 43]{Symmetries}

Теоретики Ли и Янг заключили отсутствие экспериментальных данных в отношении того, сохраняется ли чётность при слабых взаимодействиях.\cite{Lee&Yang} Для проверки этого, Цзяньсюн Ву провела эксперимент по $\beta$-распаду $^{60}\mathrm{Co}$ (реакция ${^{60}_{27}\mathrm{Co}}\to {^{60}_{28}\mathrm{Ni}} + e^- + \bar{\nu}_e + 2\gamma $) в котором сравнивались распределения направления излучения электронов и фотонов.\cite{Wu}

Излучение фотонов -- электромагнитный процесс, и следовательно -- сохраняющий чётность; к тому же, гамма лучи взаимодействуют с электронами через слабое взаимодействие. Если бы электроны всегда излучались в том же направлении и количестве что и фотоны, можно было бы заключить сохранение чётности в слабых взаимодействиях. Если же распределения электронов и фотонов были бы разными -- чётность нарушалась.

В результате эксперимента была установлена не только разница между направлениями излучений электронов и фотонов, но и то, что электроны излучались преимущественно в направлении, противоположном спину ядра: нарушение $P$-симметрии в слабых взаимодействиях максимально.
%
За экспериментальное доказательство нарушения чётности и значимый вклад в физику высоких энергий и Стандартной Модели, Ли и Янг были награждены Нобелевскуой премией по физике в 1957 году; госпожа Ву получила премию Вольфа в 1978 году.

В 1991 году, эксперимент нарушения чётности в сильных взаимодействиях проводился с поляризованным рассеиванием $pp$~\cite{Eversheim}. Нарушение было зафиксировано на уровне $10^{-7}$ -- что соответствует предсказанию нарушения симметрии засчёт (сопутствующих) слабых взаимодействий, -- при систематических ошибках на уровне $10^{-8}$. 

$P$-симметрия была первой, чьё нарушение было зафиксировано.

\subsection{C-инвариантность}

Коллаборацией WASA-at-COSY проводилось изучение нарушения $C$-симметрии в $dd$-соударениях. \cite{WASA-at-COSY_Hepi} 
%
При рассеянии дейтона на дейтоне, одной из возможных реакций является $dd \to {^4}\mathrm{He}\,\pi^0$, по наблюдению сечения которой можно судить о нарушении $C$-симметрии в сильном взаимодействии. В силу закона сохранения изотопического спина, сечение этой реакции должно быть подавлено. Поскольку $C$-симметрия является следствием закона сохранения изотопического спина, наблюдение ненулевого сечения этой реакции свидетельствует о нарушении симметрии сопряжения знака.

В эксперименте, дейтронный пучок с моментом 1.2 ГэВ/с рассеивался на замороженных дейтериевых гранулах; средствами детекторов установки WASA идентифицировались вылетающие атомы ${^4\mathrm{He}}$ (использовался передний детектор) и пары фотонов (центральный детектор), образующихся при распаде $\pi^0$-мезона. Передний детектор состоит из нескольких слоёв пластиковых сцинтилляторов, а так же из набора трубок для отслеживания трэков частиц. Частью центрального детектора, которая детектирует фотонные пары, является сцинтилляторный электромагнитный калориметр.~\cite{WASA-at-COSY-Henpi}

Проблема интерпретации полученных в эксперименте результатов (полное сечение на уровне 100 пб) заключается в схожести начальных условий исследуемой реакции, и реакции $dd \to {^3}\mathrm{He}\,n\pi^0$, которая не запрещена $C$-симметрией и является основным каналом фона для исследуемой реакции. 

\subsection{T-инвариантность}

Эксперименты по нарушению $T$-симметрии, как правило, заключаются в измерении асимметрии реакции при обращении какой-либо величины (такой как поляризация), таким образом, чтобы получившаяся реакция была эквивалентна временному обращению изначальной.

Прямым тестом нарушения $T$-симметрии в слабых взаимодействиях является эксперимент CPLEAR, в котором проводилось изучение каонных систем.~\cite{CPLEAR} Исследовалась аннигиляция $\bar{p}p \to K^+ \pi^- \bar{K}^0$ или $K^- \pi^+ K^0$. Детектируя заряженные продукты реакции, и их временную эволюцию, можно вычислить изначальную странность произведённых нейтральных каонов. После производства, каоны свободно распадались под действием слабого взаимодействия. Сравнивая количество каонов, превратившихся в антикаоны, и наоборот, можно вычислить временную асимметрию: если $T$-симметрия верна, количество превращений в одну сторону должно быть одинаково количеству превращений в другую.
%
В CPLEAR было обнаружено несовпадение количеств этих реакций, такое, что временная асимметрия вычислена с точностью $6.6\cdot 10^{-3}$.

В 2012 году, BaBar-коллаборация произвели прямое наблюдение нарушения $T$- инвариантности при деградации запутанных квантовых состояний неитральных B-мезонов, в состояния с определённым ароматом. Измеренные параметры $T$-нарушения -- $\Delta S_T^+$ и $\Delta S_T^-$, -- при предположении Гауссового распределения ошибок, соответствуют 14 стандартным отклонениям от нуля, что является статистически значимым наблюдением нарушения $T$-симметрии.~\cite{BaBar} Значение нарушения $T$-симметрии полученное в BaBar согласуется со стандартной моделью элементарных частиц и результатами CPLEAR.

\section{TRIC: истинный нуль эксперимент}

При изучении ядерных взаимодействий, существует три возможности проверки временной инвариантности~\cite{Goldstein}:
\begin{inparaenum}[i)]
	\item сравнение величин в прямой, и обращённой во времени реакциями;
	\item вместо обращённой реакции можно использовать \emph{время-сопряжённую}, то есть такую, которая обращается в прямую реакции при обращении времени;
	\item можно попытаться измерить значение величины, которая равна нулю при Т-симметрии.
\end{inparaenum}

Первые два типа экспериментов сильно ограничены в точности: в первом случае это связано с невозможностью провести прямую и обращённую реакции на одной и той же инструментальной базе, а иногда и ускорителе, в то время как во втором необходимо сравнивать разные величины. Третий тип эксперимента, хотя и коллосально превосходящий первые два в точности,\footnote{Возможный выигрыш варьируется от двух до четырёх порядков величины.} ранее считался невозможным. В 1985 году, коллаборацией трёх американских теоретиков\footnote{Фирузом Арашем, Майклом Моравксиком, и Гэри Голдстином.}, была доказана~\cite{Goldstein} невозможность однозначной интерпретации результатов эксперимента \emph{только} на основании структуры реакции и законов сохранения, без дополнительных предположений о её динамике. 

Результат теоремы следует из того, что любая наблюдаемая (билинейная относительно амплитуд вероятности)  не может обращаться в ноль \emph{в принципе}, вне зависимости от статуса временной симметрии; однако, как было отмечено авторами статьи, существует одно уникальное отношение, которое, в одном уникальном случае, связывает билинейную наблюдаемую с линейной. 

В 1990 году, на основании оптической теоремы, теоретик из национальной лаборатории Лоренца (Беркли, Калифорния), Гомер Конзетт, доказал~\cite{Conzett} существование нуль-наблюдаемых для тестов Т-инвариантности. В качестве наблюдаемой Конзетт предложил использовать величину асимметрии сечения взаимодействия между векторно-поляризованными частицами со спином $\sfrac{1}{2}$ и тензорно-поляризованными частицами со спином 1. Полное сечение рассеяния в этой реакции не нарушает P-симметрию, и не хависит от динамики процесса, таким образом состявляя основу истинного нуль-эксперимента временной инвариантности в системе барионов.

В этом и заключается физическая суть эксперимента TRIC (Time Reversal Invariance at COSY), вторая итерация которого на синхротроне COSY (в исследовательском центре города Юлих, Германия), планируется на данный момент. 

%Методология эксперимента подразумевает накапливание данных по спаду среднего тока пучка для вычисления его времени жизни. По разнице во времени жизни пучка в двух поляризационных состояниях можно судить о нарушении инвариантности: если она равна нулю --- симметрия сохраняется. 
%
%В 2006 году TRIC был проведён с использованием непрерывного протонного пучка. Особенностью использования пучка с такой временной структурой в том, что для измерения его тока нужно использовать активный трансформатор тока, частота измерений которого ограничена его электроникой, в нашем случае --- до 7 кГц.
%
%Данная итерация эксперимента будет использовать сгруппированный пучок --- т.е. переменный ток --- с частотой обращения 1 МГц. Таким образом, за одно и то же время можно накопить в 1000 раз больше данных, тем самым улучшив точность.

\pagebreak

\begin{thebibliography}{9}

\bibitem{SEP.Symmetry}
Brading, Katherine, and Elena Castellani. “Symmetry and Symmetry Breaking.” In The Stanford Encyclopedia of Philosophy, edited by Edward N. Zalta, Spring 2013., 2013. \url{http://plato.stanford.edu/archives/spr2013/entries/symmetry-breaking/}.


\bibitem{Symmetries}
Marco S. Sozzi,
\SrcNm{Discrete Symmetries and CP violation}.
University of Pisa,
2008

\bibitem{ConceptOfSpace}
Huggett, Nick, and Carl Hoefer. “Absolute and Relational Theories of Space and Motion.” In The Stanford Encyclopedia of Philosophy, edited by Edward N. Zalta, Spring 2015., 2015. \url{http://plato.stanford.edu/archives/spr2015/entries/spacetime-theories/}


\bibitem{Lee&Yang}
T. D. Lee, C. N. Yang,
\SrcNm{Question of Parity Conservation in Weak Interactions}.
Phys. Rev. 104 (1956) 254--258.

\bibitem{Wu}
C. S. Wu, E. Ambler et al.,
\SrcNm{Experimental Test of Parity Conservation in Beta Decay}.
Phys. Rev. 105 (1957) 1423--1415.

\bibitem{Eversheim}
P. D. Eversheim et al.,
\SrcNm{Parity violation in proton proton scattering at 13.6 MeV}.
Phys. Lett. B 256 (1991) 11-14.

\bibitem{WASA-at-COSY_Hepi}
P. Adlarson, W Augustyniak, W. Bardan et al.,
\SrcNm{Charge Symmetry Breaking in $dd \to {}^4He \pi^0$ with WASA-at-COSY}.
Phys. Lett. B 739 (2014) 44–49.

\bibitem{WASA-at-COSY-Henpi}
P. Adlarson, W Augustyniak, W. Bardan et al.,
\SrcNm{Investigation of the $dd \to {}^3He n \pi^0$ reaction with the FZ J\"{u}lich WASA-at-COSY facility}.
Phys. Rev. C 88 (2013) 014004.

\bibitem{CPLEAR}
A. Angelopoulos et al.,
\SrcNm{Physics at CPLEAR}. 
Phys. Rep. 374 (2003) 165--270.

\bibitem{BaBar}
J. P. Lees, V. Poireau, V. Tisserand et al.,
\SrcNm{Observation of Time-Reversal Violation in the $B^0$ Meson System}.
Phys. Rev. Lett. 109 (2012) 211801

\bibitem{Proposal}
R. Beck, P.D. Eversheim, F. Hinterberger et al.,
Cosy Proposal \# 215:
\textsc{Test of Time-Reversal Invariance in Proton-Deuteron Scattering at 
COSY}.
Helmholtz Institut für Strahlen- und Kernphysik, University Bonn, Germany.

\bibitem{Goldstein}
Arash, Firooz, Michael J. Moravcsik, and Gary R. Goldstein. “Dynamics-Independent Null Experiment for Testing Time-Reversal Invariance.” Physical Review Letters 54, no. 25 (1985): 2649.
\url{https://journals.aps.org/prl/pdf/10.1103/PhysRevLett.54.2649}

\bibitem{Conzett}
Homer E. Conzett. “On Null Tests of Time-Reversal Lnvariance,” 6. Paris, France, 1990. \url{https://publications.lbl.gov/islandora/object/ir%3A93728/datastream/PDF/download/citation.pdf}


\bibitem{COSYaccNN}
A. Lehrach, U. Bechstedt, J. Dietrich et al.,
\SrcNm{ACCELERATION OF THE POLARIZED PROTON BEAM IN THE
COOLER SYNCHROTRON COSY}.
Proc. of PAC'99, New York City, 
1999.

%\bibitem{PhDYe}
%Qiujian Ye,
%\SrcNm{Strangeness Production in Selected Proton-Induced
%Processes at COSY-ANKE}.
%Department of Physics in the Graduate School of Duke University,
%2013.
%
%\bibitem{Barlow}
%Roger Barlow
%\textsc{Systematic Errors: Facts and Fictions}.
%Department of Physics, Manchester University, England,
%2002.
%
%\bibitem{Sinervo}
%Pekka K. Sinervo
%\textsc{Definition and Treatment of Systematic Uncertainties in High Energy Physics and Astrophysics}.
%PHYSTAT2003, SLAC, Stanford, California, 
%September 8-11, 2003.
%
%\bibitem{Samuel}
%Deepak Samuel
%\textsc{Test of Feasibility of a Novel High Precision Test of Time Reversal Invariance}.
%Mathematisch-Naturwissenschaftlichen Fakult\"{a}t der Rheinischen Friedrich-Wilhelms-Universit\"{a}t Bonn,
%2007
%
%\bibitem{Ohlsen}
%Gerald G Ohlsen
%\textsc{polarization transfer and spin correlation experiments in nuclear physics}.
%Los Alamos Scientific Laboratory, University of California, Los Alamos, New Mexico 87544, 
%1972
%
%\bibitem{Barschel}
%Colin Barschel
%\textsc{calibration of the Breit-Rabi polarimeter for the PAX spin-filtering experiment at COSY/J\"ulich and AD/CERN}.
%Faculty of Mathematics, Computer Science and Natural Sciences of RWTH Aachen,
%2010.
%
%\bibitem{Gaisser}
%Martin Gai\ss er
%\textsc{Simulations of the Atomic Beam Transport in an Atomic Beam Source under the influence of Spin-Selective Sextupole Magnets}.
%Mathematisch-Naturwissenschaftlichen Fakult\"{a}t der Universit\"{a}t zu K\"oln,
%2013.
%
%\bibitem{tensors}
%Kees Dullemond \& Kasper Peeters
%\textsc{Introduction to Tensor Calculus}.
%University of Nijmegen / University of Amsterdam.
%
%\bibitem{SeasonalErrors}
%Christian Servin, Martine Ceberio, Aline James, et al.
%\textsc{How to Describe and Propagate Uncertainty When Processing Time Series: Methodological and Computational Challenges, with Potential Applications to Environmental Studies}.
%Cyber-ShARE Center, University of Texas at El Paso, El Paso, TX 79968, USA. 

\end{thebibliography}

\end{document}

