\documentclass{beamer}

\usepackage{xfrac}

\usetheme{default}

\title{NBICS Technologies}
\author{Alexander Aksentyev}
\institute{National Research Institute ``MEPhI''}
\date{}


\begin{document}
	\begin{frame}
		\titlepage
	\end{frame}

\section{Nano-technologies}

\begin{frame}
	\frametitle{Carbon allotropes}
	\begin{columns}
		\column{.4\textwidth}
		Allotropy is the property of some chemical elements to exist in several different geometries (known as \emph{allotropes}) in the same physical phase.
		\column{.6\textwidth}
		\includegraphics[scale=.45]{CarbAllotropes}
	\end{columns}
\end{frame}

\begin{frame}
	\frametitle{The Buckyball}
	\begin{columns}
		\column{.7\textwidth}
		\begin{itemize}
			\item The Buckminsterfullerene (named after inventor Richard Buckminster Fuller) was one of the first nanoparticles to be discovered (1985)
			
			\item Number of atoms: 20 to over 100; the most common type (C60) contains 60 carbon atoms
					
			\item Modifying a buckyball by adding or replacing an atom in order to change the properties of the buckyball is called \textbf{functionalization}
		\end{itemize}
		\column{.5\textwidth}
		\includegraphics[scale=.2]{buckyball_white}
	\end{columns}
\end{frame}

\begin{frame}
	\frametitle{Uses}
	\begin{itemize}
		\item \textbf{Armor}. Hard as diamonds, buckyballs are potentially useful within armor
		
		\item \textbf{Medicine}. Functionalized buckyballs can be made soluble by body cells, and hence find the following medical applications:
			\begin{itemize}
				\item As antioxidants, because of their ability to absorb electrons in free radicals
				
				\item In targeted drug delivery. The buckyball encases a minute dose of a particular drug. By controlling the functionalization of the buckyball the drug is absorbed only by the necessary cells
			\end{itemize}
		\item \textbf{Fiber optics}. Because of their perfect spherical shape, buckyballs are able to transmit light
	\end{itemize}
\end{frame}

\begin{frame}
	\frametitle{The nanotube}
	\begin{columns}
		\column{.5\textwidth}
		\begin{itemize}
			\item Diameter $<$ 1 nm
			\item A few nano- up to a millimeter in length
			\item Symmetry: armchair, zig-zag, chiral
			\item Single/multiple wall CNTs
			\item Compared to steel:
				\begin{itemize}
					\item 100 $\times$ more difficult to tear apart
					\item 5 $\times$ as elastic
					\item a quarter density
				\end{itemize}
			\item High thermal conductivity
			\item Metallic/semi-conductive contingent on symmetry
		\end{itemize}
		\column{.65\textwidth}
		\begin{minipage}{.5\textheight}
			\includegraphics[scale=.65]{nanotube_orientations}
		\end{minipage}
	\begin{minipage}{.5\textheight}
		\includegraphics[scale=.1]{MWCNT}
	\end{minipage}
	\end{columns}
\end{frame}

\begin{frame}
	\frametitle{Uses}
	\begin{enumerate}
	\item \textbf{Medicine}: functionalization, as well as their natural fluorescence, enable the use of CNTs as chemical sensors; they have also been shown to fuse well with bone, which could be used to diminish the implant rejection rate
	\item \textbf{Conductive plastics}: CNTs are the best known conductive fillers because of their high aspect ratio
	\item \textbf{Energy storage}: good battery electrodes due to high surface area ($\sim 1000$ m$^2$/g), good electrical conductivity, and linear geometry; the high surface area and thermal conductivity also make them useful as electrode catalysts in fuel cells
	\item \textbf{Molecular electronics}: their geometry, electrical conductivity, and the ability to be precisely derived, make CNTs invaluable connectors between switches at the nanoscale; their properties as semiconductors also make them usable as switches themselves
	\end{enumerate}
\end{frame}

\begin{frame}
	\frametitle{The nanobud}
	\begin{columns}
		\column{.65\textwidth}
		\begin{itemize}
			\item A nanotube with a fullerene ball attached to it
			\item As chemically reactive as the fullerenes, as electrically conductive as the nanotubes
			\item The fullerene buds serve as additional anchors, modifying the mechanical properties of the whole structure
			\item Efficient field emitters, with the emission threshold 0.65 V/$\mu$m (a third of that of the nanotubes)
			\item Highly scalable production processes, therefore applications of industrial importance
		\end{itemize}
		\column{.5\textwidth}
			\includegraphics[scale=.15]{Nanobud}
	\end{columns}
\end{frame}

\section{Bio-technologies}
\begin{frame}
	\frametitle{Synthetic biology}
	
\end{frame}

\section{Info-technologies}

\section{Cogno-technologies}

\section{Socio-technologies}

\end{document}