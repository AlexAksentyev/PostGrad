\documentclass{extarticle}
\usepackage[left=2cm,right=1.5cm,top=1.5cm, bottom=1.5cm]{geometry}

\usepackage{multirow}
\usepackage{hyperref}
\usepackage{url}

\usepackage{graphicx}
\usepackage{wrapfig}

\usepackage[utf8]{inputenc}
\usepackage[T1]{fontenc}
\usepackage[english, russian]{babel}


\begin{document}

\begin{titlepage}
	\begin{center}
		
		
		\textsc{\textbf{пояснительная записка \\
				к выпускной квалификационной работе на тему}}\\[2cm]
		
		% Title
		\textsc{Исследование влияния возможных систематических ошибок на результаты эксперимента по изучению временной инваринтности на ускорителе COSY \\[2.4cm] }
		
		
	\end{center}
	
	
	\begin{flushright}
		% Author and supervisor
		\begin{tabular}{rr}
			Студент-дипломник \underline{\hspace*{3cm}} & А.Е. Аксентьев \\
			&	группа А12-06 \\					
			Научный руководитель: Dr. rer. nat. \underline{\hspace*{3cm}} & Ю.В. Вальдау \\
			Консультант:          Доц., к.ф.-м.н. \underline{\hspace*{3cm}} & С.М. Полозов 	\\
			Рецензент:			  Доц., к.ф-м.н. \underline{\hspace*{3cm}} & А.А. Тищенко \\
			Зав. кафедрой:		  Проф., д.ф.-м.н. \underline{\hspace*{3cm}} & А.Н. Диденко
		\end{tabular}
		
	\end{flushright}
	
	\vfill
	
	
	\begin{center}
		\the\year{}
	\end{center}
	
	
	
\end{titlepage}

\tableofcontents
\pagebreak

\section*{Введение}
JEDI-коллаборация (J\"ulich Electric Dipole Moment Investigations) была создана в 2011 году с целью провести долгосрочный проект по измерению электрического дипольного момента (ЭДМ) заряженных частиц в накопительном кольце. На текущий момент, коллаборация базируется на синхротроне COSY (Institut f\"ur Kernphysik, Forschungszentrum J\"ulich, Юлих, Германия), где разрабатывает концептуальный дизайн накопительного кольца для поиска дейтронного ЭДМ.

\begin{wrapfigure}{R}{.5\textwidth}
	\includegraphics{img/cosy_220}
	\caption{COler SYnchrotron COSY в Исследовательском центре ``Юлих.''}
\end{wrapfigure}

Зачем искать ЭДМ.
	\begin{itemize}
		\item Барионная асимметрия вселенной;
		\item Условия Сахарова;
		\item Связь ЭДМ с нарушениями P-,T-симметрий
		\item Предсказания Стандартной Модели о величине ЭДМ;
		\item Если действительно найдём ЭДМ больше, чем предсказывает СМ, то значит нашли физику за её пределами.
	\end{itemize}
Поиск ЭДМ --- это мегазадача, над которой работает множество исселдовательских групп по всему миру. В частности, в JEDI коллаборации участвуют, среди прочих, исследователи из: CERN, Петербургского Института Ядерной Физики (Гатчина, Россия), Brookhaven National Laboratory (Аптон, Нью-Йорк, США), IKP Forschungszentrum J\"ulich (Юлих, Германия), LPSC (Гренобль Франция), Istituto Nazionale di Fisica Nucleare (Феррара, Италия).~\cite{JEDI}

\section{Финансирование}
В 2010 году директор Института Ядерной Физики (IKP) Исследовательского центра ``Юлих,'' профессор Ганс Штрёер получил грант от Европейского совета по научным исследованиям (European Research Council) на исследование возможности поляризации антипротонов.~\cite{ERCGrant10} Эти деньги были использованы на COSY для разработки и подтверждения работоспособности метода поляризации на протонах, чтобы в дальнейшем перенести эту методологию в CERN, на эксперименты с антипротонами.

В начале 2016 года, ERC выдал грант на пять лет (начиная с октября 2016) Юлихской группе (в лице доктора Штрёера), для дальнейших исследований в области Барионной Асимметрии вселенной.~\cite{ERCGrant16} Этот грант будет поддерживать также группы из Рейнско-Вестфальского Технического Университета Аахена (RWTH Aachen) в Германии, и Университета Феррары в Италии. Грант составляет 2.4 миллиарда евро.

\section{Структура коллаборации~\cite{ERCGrant12}}

%\begin{table}
%	\begin{tabular}{llp{4.5cm}}
%		\# & Функция     & Ключевой персонал                                    \\ \hline
%		1  & Руководство & \multirow{3}{}{H. Str\"oher (Юлих, Директор IKP-2);}
%	\end{tabular}
%\end{table}

Исполнительный совет коллаборации состоит из: Hans Str\"oher (председатель, директор IKP-2); Andreas Lerach, J\"org Pretz, Frank Rathmann (публичные представители); Volker Hejny (координатор по анализу данных); Alexander Nass (технический координатор). 

Главами исследовательских групп вне института ядерных исследований являются:
\begin{itemize}
	\item Jean Marie de Conto (Франция),
	\item Andro Kacharava (Грузия),
	\item Paolo Lenisa (Италия),
	\item Yannis Semertzidis (Корея),
	\item Andrzej Magiera (Польша),
	\item Николай Николаев (Россия),
	\item Edward Stephenson (США).
\end{itemize}

\begin{wrapfigure}{R}{.5\textwidth}
	\includegraphics[scale=.4]{img/JEDI_architecture}
	\caption{Исследовательсике группы, аффилированные с JEDI.}
\end{wrapfigure}


На сегодняшний день, коллаборация насчитывает 132 члена.

\section{Поиск ЭДМ в накопительном кольце}
Мегазадача требует мегаустановку, для своего решения. Как искать ЭДМ с помощью накопительного кольца.
	\begin{itemize}
		\item Что такое накопительное кольцо;
		\item Уравнение T-BMT;
		\item[\textbf{short:}] Условие заморозки спина => концепция Frozen Spin (BNL proposal~\cite{BNL});
		\item[\textbf{short:}] Незначительное ослабление полезного сигнала при релаксации FS условия => концепция Quasi-frozen Spin (Сеничев~\cite{Senichev});
		\item[\textbf{summary:}] Четыре фундаментальные концепции поиска ЭДМ в неидеальном накопительном кольце.
	\end{itemize}
	
\begin{figure}
	\centering
	\includegraphics[scale=.6]{TheFourConcepts.pdf}
	\caption{Четыре концепции.}
\end{figure}

\section{Накопительные кольца для поиска дейтронного ЭДМ}
\subsection{FS кольцо}
	\subsubsection{Концепция эксперимента}
		Измеряем рост вертикальной компоненты поляризации за 1000 секунд.
	\subsubsection{Структура ускорителя}
	\subsubsection{Декогеренция спина}
		Ограничивает врямя измерения => нужно достичь 1000 секунд. Методы борьбы.
	\subsubsection{Систематические ошибки}
\subsection{QFS кольцо}
	\subsubsection{Концепция эксперимента}
		Фитируем сигнал, оцениваем частоту.
	\subsubsection{Структура ускорителя}
		Два варианта структуры.
	\subsubsection{Калибровка}
		Не подавляем МДМ прецессию спина => сравнение CW/CCW частот => калибровка. Как производится калибровка магнитного поля.
		
\section*{Заключение}
На данный момент, в исследовательской программе CERN планируется пауза на десять лет [в связи с анализом данных по Хиггсу?]. В связи с этим рассматривается список задач фундаментальной физики, которыми можно было бы заняться в это время. Среди приоритетных задач этого списка --- поиск ЭДМ. 

Поскольку протонное кольцо можно сделать полностью электростатическим (позволяет величина $G$), в то время как дейтронное принципиально требует магнитные элементы, если ЦЕРН решит заняться поиском ЭДМ, вероятнее всего будет построено протонное кольцо.
\begin{thebibliography}{0}
	\bibitem{JEDIatJuelich}
	Institute for Nuclear Physics, IKP-2: Experimental Hadron Dynamics. \url{http://www.fz-juelich.de/ikp/ikp-2/EN/Forschung/JEDI/_node.html}
	\bibitem{ERCGrant10}
	Str\"oher, H. Пресс-конференция. \url{https://www.fz-juelich.de/SharedDocs/Videos/PORTAL/EN/erc/erc-grant-stroeher.html}
	\bibitem{ERCGrant16}
	Lenisa, P., Pretz, J., Str\"oher, H. (2016). Storage ring steps up search for electric dipole moments. \textit{CERN Courier}. \url{http://cerncourier.com/cws/article/cern/65816}
	\bibitem{ERCGrant12}
	Rathmann, F. Application for an ERC Advanced Grant 2012. \url{http://collaborations.fz-juelich.de/ikp/jedi/public_files/proposals/merged_document.pdf}
	\bibitem{JEDI}
	JEDI Collaboration. \url{http://collaborations.fz-juelich.de/ikp/jedi/index.shtml}
	\bibitem{BNL}
	D. Anastassopoulos, V. Anastassopoulos, D. Babusci. AGS Proposal: Search for a permanent electric dipole moment of the deuteron nucleus at the $10^{-29} e\cdot cm$ level. [Internet]. BNL; 2008 [cited 2016 Nov 25]. Available from: \url{https://www.bnl.gov/edm/files/pdf/deuteron_proposal_080423_final.pdf}
	\bibitem{Senichev}
	Yurij Senichev. Search for the Charged Particle Electric Dipole Moments in Storage Rings. In: 25th Russian Particle Accelerator Conf(RuPAC’16), St Petersburg, Russia, November 21-25, 2016 [Internet]. JACOW, Geneva, Switzerland; 2017 [cited 2017 Apr 5]. p. 6–10. Available from: \url{http://accelconf.web.cern.ch/AccelConf/rupac2016/papers/mozmh03.pdf}
	
\end{thebibliography}
\end{document}