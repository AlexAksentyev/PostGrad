\documentclass{extarticle}
\usepackage[left=2cm,right=1.5cm,top=1.5cm, bottom=1.5cm]{geometry}

\usepackage{multirow}
\usepackage{hyperref}
\usepackage{url}

\usepackage{graphicx}
\usepackage{wrapfig}

\usepackage[utf8]{inputenc}
\usepackage[T1]{fontenc}
\usepackage[english, russian]{babel}


\begin{document}

\begin{titlepage}
	\begin{center}
		
		
		\textsc{\textbf{пояснительная записка \\
				к выпускной квалификационной работе на тему}}\\[2cm]
		
		% Title
		\textsc{Исследование влияния возможных систематических ошибок на результаты эксперимента по изучению временной инваринтности на ускорителе COSY \\[2.4cm] }
		
		
	\end{center}
	
	
	\begin{flushright}
		% Author and supervisor
		\begin{tabular}{rr}
			Студент-дипломник \underline{\hspace*{3cm}} & А.Е. Аксентьев \\
			&	группа А12-06 \\					
			Научный руководитель: Dr. rer. nat. \underline{\hspace*{3cm}} & Ю.В. Вальдау \\
			Консультант:          Доц., к.ф.-м.н. \underline{\hspace*{3cm}} & С.М. Полозов 	\\
			Рецензент:			  Доц., к.ф-м.н. \underline{\hspace*{3cm}} & А.А. Тищенко \\
			Зав. кафедрой:		  Проф., д.ф.-м.н. \underline{\hspace*{3cm}} & А.Н. Диденко
		\end{tabular}
		
	\end{flushright}
	
	\vfill
	
	
	\begin{center}
		\the\year{}
	\end{center}
	
	
	
\end{titlepage}

\tableofcontents
\pagebreak

\section*{Введение}
JEDI-коллаборация (J\"ulich Electric Dipole Moment Investigations) была создана в 2011 году с целью провести долгосрочный проект по измерению электрического дипольного момента (ЭДМ) заряженных частиц в накопительном кольце (srEDM: Storage Ring EDM). На текущий момент, коллаборация базируется на синхротроне COSY (Institut f\"ur Kernphysik, Forschungszentrum J\"ulich, Юлих, Германия), где разрабатывает концептуальный дизайн накопительного кольца для поиска дейтронного ЭДМ.

Поиск Электрического Дипольного Момента (ЭДМ) в невырожденных системах был инициирован Эдвардом Пёрселлом и Норманом Рэмзи более 50 лет назад, для нейтрона. С тех пор было проведено множество всё более чувствительных экспериментов на нейтронах, атомах, и молекулах, и тем не менее, ЭДМ пока ещё не был обнаружен. 

Интерес поиска ЭДМ в том, что, если они существуют, они нарушают P- и T-симметрии. Дело в том, что вся наблюдаемая вселенная состоит преимущественно из материи; антиматерия может быть получена в ускорителях заряженных частиц, но в пренебрежимо малых количествах. На сколько мы понимаем, вскоре после Большого Взрыва, материя была образована из энергии в парах частица-античастица, после чего последовала стадия аннигиляции --- превращения пары частица-античастица обратно в энергию, --- однако по какой-то причине, эта фаза закончилась превалированием материи над антиматерией (по крайней мере в наблюдаемой вселенной) --- процесс называемый \emph{бариогенезом.}

В 1967 году, академик АН СССР Андрей Сахаров определил условия, требуемые для бариогенеза (независимо от механизма его действия). Одно из \emph{условий Сахарова} --- существование процессов, нарушающих C- и CP-симметрии. Известны источники нарушения этих симметрий, однако их не достаточно для объяснения барионной асимметрии вселенной; поиск продолжается.

Ненулевые ЭДМ элементарных частиц могут привести нас к физике за границами Стандартной Модели элементарных частиц; такие теории как SUSY (суперсимметрия) указывают на наличие ЭДМ гораздо больше, чем предсказывает Стандартная Модель элементарных частиц.

Проект srEDM --- это сложное, high-risk high-impact предприятие, нуждающееся в тщательном планировании и исполнении. В следующих разделах рассмотрены: структура необходимой команды экспертов, требующееся оборудование, а также этапы развития исследовательской программы.


\section{Структура команды и партнёры}

Не смотря на то, что Институт ядерных исследований, включающий в себя отдел ускорителей, а также центральные инженерные институты, является крепким фундаментом проекта, разнообразность и сложность поставленной задачи требуют дополнительной экспертизы, которая предоставляется двумя институтами-партнёрами:
\begin{itemize}
	\item Институт физики Аахенского университета
\end{itemize}

		
\section*{Заключение}
На данный момент, в исследовательской программе CERN планируется пауза на десять лет [в связи с анализом данных по Хиггсу?]. В связи с этим рассматривается список задач фундаментальной физики, которыми можно было бы заняться в это время. Среди приоритетных задач этого списка --- поиск ЭДМ. 

Поскольку протонное кольцо можно сделать полностью электростатическим (позволяет величина $G$), в то время как дейтронное принципиально требует магнитные элементы, если ЦЕРН решит заняться поиском ЭДМ, вероятнее всего будет построено протонное кольцо.
\begin{thebibliography}{0}
	\bibitem{JEDIatJuelich}
	Institute for Nuclear Physics, IKP-2: Experimental Hadron Dynamics. \url{http://www.fz-juelich.de/ikp/ikp-2/EN/Forschung/JEDI/_node.html}
	\bibitem{ERCGrant10}
	Str\"oher, H. Пресс-конференция. \url{https://www.fz-juelich.de/SharedDocs/Videos/PORTAL/EN/erc/erc-grant-stroeher.html}
	\bibitem{ERCGrant16}
	Lenisa, P., Pretz, J., Str\"oher, H. (2016). Storage ring steps up search for electric dipole moments. \textit{CERN Courier}. \url{http://cerncourier.com/cws/article/cern/65816}
	\bibitem{ERCGrant12}
	Rathmann, F. Application for an ERC Advanced Grant 2012. \url{http://collaborations.fz-juelich.de/ikp/jedi/public_files/proposals/merged_document.pdf}
	\bibitem{ERCGrant15}
	Str\"oher, H., Search for Electric Dipole Moments using Storage Rings, Horizon 2020 proposal, Excellence Science Call: ERC-2015-AdG. \url{http://collaborations.fz-juelich.de/ikp/jedi/public_files/proposals/Proposal-SEP-210276270.pdf}
	\bibitem{JEDI}
	JEDI Collaboration. \url{http://collaborations.fz-juelich.de/ikp/jedi/index.shtml}
	\bibitem{BNL}
	D. Anastassopoulos, V. Anastassopoulos, D. Babusci. AGS Proposal: Search for a permanent electric dipole moment of the deuteron nucleus at the $10^{-29} e\cdot cm$ level. [Internet]. BNL; 2008 [cited 2016 Nov 25]. Available from: \url{https://www.bnl.gov/edm/files/pdf/deuteron_proposal_080423_final.pdf}
	\bibitem{Senichev}
	Yurij Senichev. Search for the Charged Particle Electric Dipole Moments in Storage Rings. In: 25th Russian Particle Accelerator Conf(RuPAC’16), St Petersburg, Russia, November 21-25, 2016 [Internet]. JACOW, Geneva, Switzerland; 2017 [cited 2017 Apr 5]. p. 6–10. Available from: \url{http://accelconf.web.cern.ch/AccelConf/rupac2016/papers/mozmh03.pdf}
	
\end{thebibliography}
\end{document}