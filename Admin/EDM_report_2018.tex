\documentclass{article}

\usepackage{RepSty}
\usepackage{SPINDefs}
\usepackage{hyperref}

\usepackage{setspace}
\usepackage{multirow}
\usepackage{threeparttable}
\usepackage{paralist}

\newcommand{\bld}{\boldsymbol}

\begin{document}
	\singlespacing
	\begin{titlepage}

\begin{center}
{МИНИСТЕРСТВО ОБРАЗОВАНИЯ И НАУКИ РОССИЙСКОЙ ФЕДЕРАЦИИ}\\[3pt]
\textsc{\small{Федеральное государственное автономное образовательное учреждение высшего образования}}\\

\textbf{\enquote{Национальный исследовательский ядерный университет\\
{``МИФИ''}}}\\
\textbf{(НИЯУ МИФИ)}\\[2cm]




\textsc{\textbf{Отчет о научно-исследовательской деятельности\\		
		аспиранта и подготовке научно-квалификационной\\	
		работы (диссертации) на соискание ученой степени\\		
		кандидата наук за торое полугодие 3 курса}}\\[2cm]

% Title
\enquote{Изучение магнитооптических структур в системах Замороженного и Квази-замороженного спина}\\[2cm]


\end{center}


\begin{flushleft}
% Author and supervisor
\begin{tabular}{ll}
Аспирант 						& А.Е. Аксентьев \\
Направление                     & 03.06.01 Физика и астрономия \\					
Научная специальность		   	& 01.04.20 Физика пучков заряженных частиц\\
								& \-\hspace{1.8cm} и ускорительная техника \\[1cm]
Научный руководитель 			& \\
Должность, степень, звание 		& С.М. Полозов, к.ф.-м.н, доц. \\[1cm]
Дата защиты:					& \\
Результат защиты:				& \\
\end{tabular}

\end{flushleft}

\vfill


\begin{center}
Москва \the\year{}
\end{center}



\end{titlepage}
	
	\tableofcontents 
	\pagebreak
	
	\onehalfspacing
	
	\section*{Физическая мотивация}
	 Вся наблюдаемая вселенная состоит преимущественно из материи; антиматерия может быть получена в ускорителях заряженных частиц, но в пренебрежимо малых количествах. На сегодняшний день считается, что вскоре после Большого Взрыва материя была образована из энергии в парах частица-античастица, после чего последовала стадия аннигиляции; однако, по какой-то причине, эта фаза закончилась превалированием материи над антиматерией (по крайней мере в наблюдаемой вселенной) --- процесс называемый \emph{бариогенезом.}
	
	В 1967 году, академик АН СССР Андрей Сахаров определил условия, требуемые для бариогенеза (независимо от механизма его действия). Одно из \emph{условий Сахарова} --- существование процессов, нарушающих C- и CP-симметрии. Известны источники нарушения этих симметрий, однако их не достаточно для объяснения барионной асимметрии вселенной; поиск продолжается.
	
	Интерес поиска Электрического Дипольного Момента (ЭДМ) элементарных частиц состоит в том, что, если они существуют, то они нарушают P- и T-симметрии. Таким образом, обнаружение ненулевых ЭДМ элементарных частиц может привести нас к физике за границами Стандартной Модели; такие теории как SUSY (суперсимметрия) указывают на наличие ЭДМ гораздо большей величины (на уровне $10^{-29} - 10^{-24}$ e$\cdot$cm), чем предсказывает Стандартная Модель.
		
	\section{Состояние изучаемой проблемы}
	Поиск ЭДМ в невырожденных системах был инициирован Эдвардом Пёрселлом и Норманом Рэмзи более 50 лет назад, для нейтрона. С тех пор было проведено множество всё более чувствительных экспериментов на нейтронах, атомах, и молекулах, и тем не менее, ЭДМ пока ещё не был обнаружен. На данный момент, верхний предел ЭДМ нейтрона оценивается на уровне $<3\cdot 10^{-26}$ e$\cdot$cm, протона --- $<8\cdot 10^{-25}$ e$\cdot$cm.~\cite{Pretz_presentation}
	
	В 2008 году коллаборацией в Брукхейвенской Национальной Лаборатории (BNL, США) был предложен эксперимент по измерению ЭДМ дейтрона, основанный на использовании эффекта замороженного спина в магнитном накопительном кольце.~\cite{BNL} 
	
	В 2015 году, коллаборацией Storage Ring EDM Collaboration, был предложен эксперимент по поиску протонного ЭДМ в полностью электрическом накопительном кольце.~\cite{Proton_EDM}
	
	На данный момент, коллаборацией JEDI (Исследовательский центр ``Юлих,'' Германия) ведётся разработка структуры накопительного кольца для проведения предварительного эксперимента по измерению дейтронного ЭДМ на синхротроне COSY.
	
	\subsection{Предложение BNL}
	Движение спина частицы в электромагнитном поле описывается уравнением Томаса-Баргманна-Мишеля-Телегди:
	\begin{equation*}
	\begin{cases}
	\sfrac{\td\bld S}{\td t} &=\bld S \times\bld \Omega,\\
	\bld \Omega &= -\frac em \bkt*{G\bld B + \bkt{\frac 1 {\gamma^2 - 1} - G}\bld \beta \times\bld E + \frac\eta2 \bkt{\bld E +\bld\beta\times\bld B}}.
	\end{cases}
	\end{equation*}
	Здесь, $G = (g-2)/2$ --- аномальный магнитный момент; для дейтрона $G = -0.142$. Подобрав энергию частицы, можно ``заморозить'' направление её спина относительно импульса; это называется условием \emph{замороженного спина}.
	
	В концепции эксперимента, предложенного BNL~\cite{BNL}, продольно поляризованный пучок инжектируется в накопительное кольцо. При соблюдении условия замороженного спина, вектор поляризации пучка в горизонтальной плоскости сонаправлен с вектором импульса в любой момент времени. Засчёт присутствия радиальной компоненты электрического поля, при условии отличного от нуля ЭДМ, спин будет медленно поворачиваться в вертикальной плоскости. Измеряя вертикальную компоненту поляризации по истечении значительного промежутка времени, можно вычислить $\eta$. 
	
	Такая методология ограничена, в первую очередь, конечным временем когеренции спина (SCT). Декогеренция спина обусловлена различием длин орбит частиц в пучке, что в свою очередь есть результат конечности фазового объема пучка. Раскладывая выражение для спин-тюна частицы в электро-мангитном поле вокруг референсного значения получим:~\cite{SenichevRuPAC2016}
	\begin{equation*}
		\begin{cases}
		\Delta\nu_s^E &= G_6\Delta\gamma + O(\Delta\gamma),\\
		\Delta\nu_s^B &= G\Delta\gamma,\\
		\Delta\gamma(t) &= \Delta\gamma_m\cos(\Omega_s t) + f(\alpha_1, \eta, \gamma)\Delta\gamma_m^2 + g\bkt{\eta, \beta, \frac{\Delta L}{L}}\gamma^2.
		\end{cases}
	\end{equation*}
	Здесь: $G_6$ есть некоторая функция $(G, \gamma_0)$, $\Delta\gamma_m, \Omega_s$ --- амплитуда и частота синхроторонных колебаний, $\alpha_1$ коэффициент сжатия орбиты , $\eta$ --- слип-фактор, $\Delta L/L$ --- удлинение орбиты засчёт бетатронных колебаний.
	
	Первый член $\Delta\gamma(t)$ усредняется в ноль засчёт синхротронных колебаний; используя секступольные магниты можно уменьшить декогеренцию, связанную с удлинением орбиты.
	
	
	\subsection{Структуры FS и QFS колец}
	Полезный ЭДМ сигнал максимален при соблюдении условия замороженного спина; однако, это условие требует чтобы у частицы было конкретное ``магическое'' значение энергии. Таким образом, это условие не может быть выполнено для всех частиц в пучке. На практике, это условие выполняется только для референсной частицы. Также, в замороженном (FS) кольце требуется использование комбинированных E+B элементов в арках, что усложняет конструкцию ускорителя. По этим причинам, была предложена концепция кольца с квази-замороженным спином (QFS).
	
	В QFS кольце не требуется непрерывное выполнение FS условия; спин частицы возвращается в ``замороженное'' положение специальной оптикой через каждые $n$ оборотов. При прецессии спина в горизонтальной плоскости с амплитудой $\Phi_s$, рост ЭДМ-сигнала спадает пропорционально $J_0(\Phi_s) \approx 1 - \bkt{\frac{\Phi_s}{2}}^2$. Таким образом, для дейтронного кольца с $G = -0.142$, при амплитуде прецессии $\Phi_s = \pi\cdot\sfrac{\gamma G}{2n}$, деградация сигнала не превышает нескольких процентов. Также, эта структура может быть выполнена в двух вариантах: 
	\begin{inparaenum}[1)]
		\item с искривлёнными прямыми секциями с цилиндрическими электростатическими дефлекторами,
		\item с прямыми комбинированными E+B дефлекторами в прямых секциях.
	\end{inparaenum}~\cite{SenichevICAP15}

	\begin{figure}
		\centering
		\includegraphics[scale=.5]{edm_img/EB_QFS_Sx_vs_s_1turn}
		\caption{Эволюция радиальной компоненты спина частиц в QFS структуре на длине одного оборота. Вертикальные пунктирные линии обозначают границы секций ускорителя.}
	\end{figure}
	
	\section{Цели и задачи работы}
	Целью данной работы является сравнение структур FS и QFS для определения которая из них больше подходит для проведения эксперимента по поиску ЭДМ дейтрона с точностью $10^{-29}$ e$\cdot$cm.
	
	Для этого предложена следующая программа:
	\begin{enumerate}
		\item Изучение эффекта неточности установки E+B элементов (сохраняется сила Лоренца) в FS-структуре на спин-динамику пучка;
		\item То же самое для квадрупольных магнитов (сила Лоренца не сохраняется);
		\item Изучение зависимости декогеренции спина от начального распределения пучка по поперечным координатаам и энергии;
		\item Изучение оптимальной расстановки секступольных магнитов с целью подавления декогеренции и хроматичности;
		\item Моделирование калибровки магнитного поля в ускорителе по параметру $\gamma_{eff}$ в горизонтальной плоскости для процедуры CW/CCW.
	\end{enumerate}

	\section{Текущий статус}
	\subsection{Трекинговый код}
	Для проведения анализа был написан код на языке Python. В данном коде используется поэлементное интегрирование системы уравнений описанных в~\cite[стр. 39]{Ivanov_thesis}. Наклон элементов имплементирован в двух вариациях:
	\begin{inparaenum}[1)]
		\item путём вычисления матрицы 
	\end{inparaenum}
	\subsection{Маппинговый код}
	\subsection{Результаты моделирования}
	
	\clearpage
	\begin{thebibliography}{99}
		\bibitem{Pretz_presentation}
		Pretz J 2012 Measurement of Permanent Electric Dipole Moments of Proton, Deuteron, and Light Nuclei in Storage Rings. Презентация. \url{http://collaborations.fz-juelich.de/ikp/jedi/public_files/usual_event/2012-06-18_J.Pretz_SSP2012.pdf}
		
		\bibitem{BNL}
		Anastassopoulos D et al. 2008 AGS Proposal: Search for a permanent electric dipole moment of the deuteron nucleus at the $10^{-29}$ e$\cdot$cm level. Отчёт BNL. \url{https://www.bnl.gov/edm/files/pdf/deuteron_proposal_080423_final.pdf}
		
		\bibitem{Proton_EDM}
		Anastassopoulos D et al. 2015 A Storage Ring Experiment to Detect a Proton Electric Dipole Moment. \url{http://arxiv.org/abs/1502.04317}
		
		\bibitem{SenichevRuPAC2016}
		Senichev Y 2016 Search for the Charged Particle Electric Dipole Moments in Storage Rings. Proc. RuPAC2016 (St. Petersburg, Russia).

		\bibitem{SenichevICAP15}
		Senichev Y 2015 Investigation of lattice for deuteron EDM ring. Proc. ICAP2015 (Shanghai, China).
		
		\bibitem{Ivanov_thesis}
		Иванов А Нелинейное матричное интегрирование спин-орбитальной динамики заряженных частиц. СПбГУ; 2015.
	\end{thebibliography}
\end{document}