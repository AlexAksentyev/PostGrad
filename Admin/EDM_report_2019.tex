\documentclass{report}

\usepackage{RepSty}
\usepackage{hyperref}

\usepackage{setspace}
\usepackage{multirow}
\usepackage{threeparttable}
\usepackage{paralist}

\usepackage[numbers]{natbib}

\newcommand{\bld}{\boldsymbol}
\newcommand{\Wx}{\Omega_x}
\newcommand{\Wy}{\Omega_y}
\newcommand{\Dt}{\Delta t}

\newcommand{\Artem}{/home/alexa/REPOS/COSYINF/img/Artem}
\newcommand{\multisext}{\Artem/multisext_test}
\newcommand{\compare}{\Artem/spin_vs_polarization_fit_comp}
\newcommand{\decoh}{\Artem/decoherence_frequency_dependence}

\begin{document}
\singlespacing
\begin{titlepage}

\begin{center}
{МИНИСТЕРСТВО ОБРАЗОВАНИЯ И НАУКИ РОССИЙСКОЙ ФЕДЕРАЦИИ}\\[3pt]
\textsc{\small{Федеральное государственное автономное образовательное учреждение высшего образования}}\\

\textbf{\enquote{Национальный исследовательский ядерный университет\\
{``МИФИ''}}}\\
\textbf{(НИЯУ МИФИ)}\\[2cm]




\textsc{\textbf{Отчет о научно-исследовательской деятельности\\		
		аспиранта и подготовке научно-квалификационной\\	
		работы (диссертации) на соискание ученой степени\\		
		кандидата наук за первое полугодие 4 курса}}\\[2cm]

% Title
\enquote{Исследование магнитооптических структур со свойствами замороженного и квази-замороженного спина для поиска электрического дипольного момента дейтрона в накопительном кольце}\\[2cm]


\end{center}


\begin{flushleft}
% Author and supervisor
\begin{tabular}{ll}
	Аспирант                   & А.Е. Аксентьев                           \\
	Направление                & 03.06.01 Физика и астрономия             \\
	Научная специальность      & 01.04.20 Физика пучков заряженных частиц \\
	                           & \-\hspace{1.8cm} и ускорительная техника \\[1cm]
	Научный руководитель       &                                          \\
	Должность, степень, звание & С.М. Полозов, к.ф.-м.н, доц.             \\%[1cm]
	                           & Ю.В. Сеничев, д.ф-м.н., проф.            \\[1cm]
	Дата защиты:               &                                          \\
	Результат защиты:          &
\end{tabular}

\end{flushleft}

\vfill


\begin{center}
Москва \the\year{}
\end{center}



\end{titlepage}


\tableofcontents 
\pagebreak

\onehalfspacing

\chapter{Постановка целей и задач исследования}

\section{Цели, задачи, перспективы исследования}\label{sec:research_program}

Целью исследования является сравнение структур с Замороженным и Квази-замороженным спином, для определения которая из них больше подходит для проведения эксперимента по поиску ЭДМ дейтрона с точностью $10^{-29}$ e$\cdot$cm.

Для достижения поставленной цели предложена следующая программа:
\begin{enumerate}
\item Анализ требований к эксперименту по поиску дейтронного ЭДМ методом определения частоты прецессии спина, обоснованных статистическими соображениями;
\item Изучение влияния неточности установки E+B спин-ротаторов на систематическую ошибку определения частоты ЭДМ прецессии (фальш-сигнал);
\item Изучение явления деполяризации пучка засчёт эффекта декогеренции спин-тюна, а также его подавление при помощи секступольных полей в неидеальной структуре;
\item Изучение влияния нарушения спин-прецессии на конечную статистику в частотном методе определения ЭДМ;
\item Моделирование процедуры калибровки ведущего магнитного поля кольца для его точного обращения при обратной инжекции пучка в процедуре CW/CCW.
\end{enumerate}

В перспективе предполагается использование предложенного метода поиска ЭДМ элементарной частицы в частотной области, а также разработанной в рамках данной работы методологии анализа колец, при конструировании кольца по поиску дейтронного ЭДМ на комплексе НИКА, Дубна.

\section{Актуальность и научная новизна работы}

В данной работе исследуется возможность измерения дейтронного ЭДМ в накопительном кольце, построенном на основе принципа Замороженного (или Квази-замороженного) спина, методом Frequency Domain.

Поиск ЭДМ в невырожденных системах был инициирован Эдвардом Пёрселлом и Норманом Рэмзи более 50 лет назад, для нейтрона. С тех пор было проведено множество всё более чувствительных экспериментов на нейтронах, атомах, и молекулах, и тем не менее, ЭДМ пока ещё не был обнаружен. На данный момент, верхний предел ЭДМ нейтрона оценивается на уровне $<3\cdot 10^{-26}$ e$\cdot$cm, протона --- $<8\cdot 10^{-25}$ e$\cdot$cm.~\cite{Pretz_presentation}

В 2008 году коллаборацией в Брукхейвенской Национальной Лаборатории (BNL, США) был предложен эксперимент по измерению ЭДМ дейтрона, основанный на использовании эффекта замороженного спина в магнитном накопительном кольце.~\cite{BNL} 

В 2015 году, коллаборацией Storage Ring EDM Collaboration, был предложен эксперимент по поиску протонного ЭДМ в полностью электрическом накопительном кольце.~\cite{Proton_EDM}

На данный момент, коллаборацией JEDI (Исследовательский центр ``Юлих,'' Германия) ведётся разработка структуры накопительного кольца для проведения предварительного эксперимента по измерению дейтронного ЭДМ на полностью магнитном синхротроне COSY. В рамках исследований, проводимых коллаборацией, была выдвинута концепция квази-замороженного кольца. Также, профессором Ю. Сеничевым была предложена методология поиска ЭДМ в частотной области (так называемый Frequency Domain Method).

Frequency Domain Method базируется на концепциях замороженного спина, предложенного в BNL, и Spin Wheel, предложенного профессором И. Коопом (Новосибирский Государственный Университет). Однако ни тот, ни другой метод сам по себе не решают проблему неточности установки элементов ускорителя, и возникающий в связи с этим фальш-сигнал.  На сегодняшний день, Frequency Domain --- это единственный известный, практически-реализуемый метод, решающий данную проблему.

\section{Структура работы}
Структура работы следует пунктам программы обозначенной в разделе~\ref{sec:research_program}

\chapter{Характеристика современного состояния изучаемой проблемы}
Предварительный библиографический список по теме диссертации.
Диссертация Еремея Валетова.~\citep[p.~235]{Eremey:Thesis}
Диссертация Андрея Иванова.
Диссертация Станислава Чекменева.
AGS Proposal (deuteron).
Senichev: Investigation of lattice for deuteron EDM ring.
Senichev: Spin tune decoherence effects in electro- and magnetostatic structures.
Senichev: Search for the charged particle electric dipole moments in storage rings.
Senichev: Frequency domain method of the search for the deuteron electric dipole moment in a storage ring with imperfections.
JEDI: Spin Tune mapping.


\chapter{Результаты разработки основных направлений теоретической концепции (FS и QFS колец)}
\section{Статистический анализ требований к эксперименту по поиску ЭДМ}

\section{Систематическая ошибка в эксперименте, связанная с неточностью установки магнитов}

\section{Декогеренция спина в идеальном накопительном кольце}
\subsection{Требования ко времени когеренции пучка}
1000 секунд.
\subsection{Происхождение декогеренции}
\subsubsection{Принцип автофазировки}
такой вот принцип.
\subsubsection{Удлинение орбиты}
из-за принципа автофазировки.
\subsubsection{Использование ВЧ резонатора для подавления декогеренции}
ВЧ всё усредняет.
\subsubsection{Удлинение орбиты засчёт бетатронных колебаний}
потому что не на референсной орбите.
\subsubsection{Сдвиг равновесного значения импульса частицы}
дальше бежать = быстрее бежать.
\subsubsection{Понятие эффективного Лоренц-фактора}
которое всё объясняет.

\subsection{Подавление декогеренции с помощью секступольных полей}
\subsubsection{Влияние на коэффициент сжатия орбиты}
существует.
\subsubsection{Влияние на длину орбиты}
существует.
\subsection{Оптимальная расстановка секступолей}
там где максимумы функций Твисса.

\section{Смена полярности ведущего магнитного поля}
большая проблема.

\chapter{Результаты численного моделирования спин-орбитальной динамики частиц в FS (и QFS) кольцах}

\section{Фальш-сигнал, связанный с неточностью установки магнитов}
линейно зависит от среднего значения наклона элементов, и не зависит от конкретной последовательности наклонов.
\section{Декогеренция}

\subsection{Симуляция эффекта подавления декогеренции спина в вертикальной плоскости при помощи секступолей}

\begin{figure}[H]
  \centering
  \begin{subfigure}[b]{\textwidth}
    \includegraphics[width=\linewidth]{\multisext/ny_vs_offset}
    \caption{Vertical component of the SPA $\bar n_y$.\label{fig:DECOH_full_ny}}
  \end{subfigure}

  \begin{subfigure}[b]{\textwidth}
    \includegraphics[width=\linewidth]{\multisext/ny_vs_offset_zoom}
    \caption{Zoom of Figure~\ref{fig:DECOH_full_ny}. Vertical component $\bar n_y$ (and $\bar n_x$) is parabolic around the reference orbit for the optimal setting, unlike $nu_s$, which is linear.}
  \end{subfigure}
  
  \begin{subfigure}[b]{\textwidth}
    \includegraphics[width=\linewidth]{\multisext/spin_tune_vs_offset}
    \caption{Spin tune $\nu_s$.}
  \end{subfigure}
  \caption{DECOH data poltted against the beam offset in the vertical direction for each sextupole setting.}
\end{figure}

\begin{figure}[H]
  \centering
  \begin{subfigure}[b]{\textwidth}
    \includegraphics[width=\linewidth]{\multisext/FreqY_vs_offset}
    \caption{Full range.\label{fig:FreqY_vs_offset}}
  \end{subfigure}

  \begin{subfigure}[b]{\textwidth}
    \includegraphics[width=\linewidth]{\multisext/FreqY_vs_offset_zoom}
    \caption{Zoom of Figure~\ref{fig:FreqY_vs_offset}. Vertical polarization depends on the beam offset linearly, like spin tune $\nu_s$, and unlike $\bar n_y$.}
  \end{subfigure}
  \caption{Vertical polarization precession frequency estimate vs beam offset for the optimal sextupole setting (orange), and two settings at the opposite range of the swept gradient space.}
\end{figure}

\subsection{Исследование зависимости оценки частоты прецессии поляризации банча от спин тюна и прецессии оси стабильного спина}

\begin{figure}[H]
  \centering
  \includegraphics[width=\linewidth]{\decoh/ny_vs_turn}
  \caption{Vertical component of $\bar n$ for particles with offsets, resp.: [1.02749, 1.02937, 1.02840] mm. We obserwe rapid oscillations about some average level. This average level will change parabolically with the particle's vertical offset (see Figure~\ref{fig:mean_tune_axis} below). The rapid oscillations are due to betatron motion (see Figures~\ref{fig:tune_axis_position_y},~\ref{fig:tune_axis_position_x}).\label{fig:ny_vs_turn}}
\end{figure}


\begin{figure}[H]
  \centering
  \begin{subfigure}[b]{\textwidth}
    \includegraphics[width=\linewidth]{\decoh/ny_vs_y}
    \caption{Vertical $\bar n$ component vs the particle vertical position.}
  \end{subfigure}

  \begin{subfigure}[b]{\textwidth}
    \includegraphics[width=\linewidth]{\decoh/spin_tune_vs_y}
    \caption{Spin tune vs the particle vertical position.}
  \end{subfigure}
  \caption{Particle precession frequency vs its vertical offset. The plots exhibit non-functional dependence pf the parameters on the vertical particle position as a result of their dependence on the radial position, which also oscillates at a small amplitude (see Figure~\ref{fig:tune_axis_position_x}). \label{fig:tune_axis_position_y}}
\end{figure}

\begin{figure}[H]
  \centering
  \begin{subfigure}[b]{\textwidth}
    \includegraphics[width=\linewidth]{\decoh/ny_vs_x}
    \caption{Vertical $\bar n$ component vs the particle horizontal position.}
  \end{subfigure}

  \begin{subfigure}[b]{\textwidth}
    \includegraphics[width=\linewidth]{\decoh/spin_tune_vs_x}
    \caption{Spin tune vs the particle horizontal position.}
  \end{subfigure}
  \caption{Particle precession frequency vs its horizontal offset.\label{fig:tune_axis_position_x}}
\end{figure}


\section{Калибровка величины ведущего магнитного поля с помощью частоты прецессии поляризации пучка в горизонтальной плоскости}

\chapter{Результаты обобщения и систематизации результатов проведённых исследований}
\section{Связь между спин тюном и осью стабильного спина}
Концепция эффективного Лоренц-фактора предполагает равенство скоростей вращения спинов частиц с равными эффективными Лоренц-факторами вокруг их собственных осей стабильного спина. При этом известно, что ось стабильного спина частицы, участвующей в бетатронном движении, меняет свою ориентацию в пространстве от оборота к обороту. Последний факт послужил мотивацией проведения исследования влияния нарушения спин-прецессии на конечную статистику.

Была обнаружена систематическая структура оценки ошибки измерения, что свидетельствует о несоответствии используемой при фитировании данных модели функции-генератору данных. Также было сделано связанное с предыдущим наблюдение, что оценка частоты колебаний вертикальной компоненты поляризации пучка зависит от его вертикального сдвига по отношению к референсной орбите. Было выдвинуто две гипотезы о причине данного наблюдения:
\begin{inparaenum}[\itshape i\upshape)]
\item систематичность ошибки есть следствие различия скоростей прецессии частиц в пучке, и не связано с различием направлений их осей прецессий;
\item структура ошибки есть прямое следствие прецессии осей стабильного спина частиц в пучке.
\end{inparaenum}

Первая гипотеза основана на том факте, что, исходя из решения уравнения Т-БМТ, частота колебаний компоненты спина зависит только от величины вектора частоты, но не его направления; от последнего зависит только амплитуда колебаний. Соответственно, при выравнивании величины спин тюна при помощи секступольных полей, этот эффект должен спадать.

Вторая гипотеза основана на том факте, что статистическая оценка частоты не тождествена самой частоте колебаний компоненты, и подвержена влиянию модуляции амплитуды колебаний, вызываемой прецессией оси стабильного спина. К тому же, вариация компонент оси стабильного спина по частицам в анализируемой симуляции на много (8) порядков превосходит вариацию спин тюна, и потому вторая гипотеза является более вероятным объяснением наблюдаемого эффекта.

В процессе анализа данных было выяснено следующее:
\begin{itemize}
\item 
\item
\end{itemize}

\chapter{Оценка достоверности и достаточности данных исследования}

\chapter{Заключения и выводы}

\bibliography{/home/alexa/REPOS/EDM/Reports/PhD/PhDRefs}
\bibliographystyle{vancouver}

\end{document}
