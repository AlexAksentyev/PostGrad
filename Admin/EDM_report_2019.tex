\documentclass{report}

\usepackage{phdstyle}
\usepackage{repsty}

\usepackage{setspace}

\usepackage[numbers]{natbib}

\newcommand{\bld}{\boldsymbol}
\newcommand{\Wx}{\Omega_x}
\newcommand{\Wy}{\Omega_y}
\newcommand{\Dt}{\Delta t}

% % Statistical requirements definitions
\newcommand{\td}{\mathrm{d}}
\newcommand{\pars}{\boldsymbol{\theta}}
\DeclareDocumentCommand{\Fisher}{O{}D(){\pars_0}}{I_{#1}(#2)}
\DeclareDocumentCommand{\Xpct}{mo}{\mathrm{E}\bkt*{#1\IfNoValueTF{#2}{}{|~ #2}}}
\DeclareDocumentCommand{\XpctO}{m}{\Xpct{#1}[\pars_0]}
\DeclareDocumentCommand{\mupp}{s}{\mu'_\phi\IfBooleanTF{#1}{}{(t_i)}}
\DeclareDocumentCommand{\mudpp}{s}{\mu''_{\phi^2}\IfBooleanTF{#1}{}{(t_i)}}
\newcommand{\vp}[2]{{#1}\cdot 10^{#2}}
\newcommand{\cnt}{c}
\newcommand{\meas}{\epsilon}
\newcommand{\dt}{\Delta t}
\newcommand{\dtm}{\dt_{\meas}}
\newcommand{\dtc}{\dt_{\cnt}}
\newcommand{\Ncm}{{n_{\sfrac{\cnt}{\meas}}}}
\newcommand{\Nmnd}{{n_{\sfrac{\meas}{zc}}}}
\newcommand{\Nnd}{{n_{zc}}}
\newcommand{\Nm}{{n_{\meas}}}
\newcommand{\Ncnt}{{n_{\cnt}}}
\newcommand{\LTb}{\tau_b}
\newcommand{\LTd}{\tau_d}
\newcommand{\lamb}{\lambda_b}
\newcommand{\lamd}{\lambda_d}

\newcommand{\Artem}{/Users/alexaksentyev/REPOS/COSYINF/img/Artem}
\newcommand{\multisext}{\Artem/multisext_test}
\newcommand{\compare}{\Artem/spin_vs_polarization_fit_comp}
\newcommand{\decoh}{\Artem/decoherence_frequency_dependence}

\begin{document}
\singlespacing
\begin{titlepage}

\begin{center}
{МИНИСТЕРСТВО ОБРАЗОВАНИЯ И НАУКИ РОССИЙСКОЙ ФЕДЕРАЦИИ}\\[3pt]
\textsc{\small{Федеральное государственное автономное образовательное учреждение высшего образования}}\\

\textbf{\enquote{Национальный исследовательский ядерный университет\\
{``МИФИ''}}}\\
\textbf{(НИЯУ МИФИ)}\\[2cm]




\textsc{\textbf{Отчет о научно-исследовательской деятельности\\		
		аспиранта и подготовке научно-квалификационной\\	
		работы (диссертации) на соискание ученой степени\\		
		кандидата наук за первое полугодие 4 курса}}\\[2cm]

% Title
\enquote{Исследование магнитооптических структур со свойствами замороженного и квази-замороженного спина для поиска электрического дипольного момента дейтрона в накопительном кольце}\\[2cm]


\end{center}


\begin{flushleft}
% Author and supervisor
\begin{tabular}{ll}
	Аспирант                   & А.Е. Аксентьев                           \\
	Направление                & 03.06.01 Физика и астрономия             \\
	Научная специальность      & 01.04.20 Физика пучков заряженных частиц \\
	                           & \-\hspace{1.8cm} и ускорительная техника \\[1cm]
	Научный руководитель       &                                          \\
	Должность, степень, звание & С.М. Полозов, к.ф.-м.н, доц.             \\%[1cm]
	                           & Ю.В. Сеничев, д.ф-м.н., проф.            \\[1cm]
	Дата защиты:               &                                          \\
	Результат защиты:          &
\end{tabular}

\end{flushleft}

\vfill


\begin{center}
Москва \the\year{}
\end{center}



\end{titlepage}


\tableofcontents 
\pagebreak

\onehalfspacing

\chapter{Постановка целей и задач исследования}

\section{Цели, задачи, перспективы исследования}\label{sec:research_program}

Целью исследования является сравнение структур с Замороженным и Квази-замороженным спином, для определения которая из них больше подходит для проведения эксперимента по поиску ЭДМ дейтрона с точностью $10^{-29}$ e$\cdot$cm.

Для достижения поставленной цели предложена следующая программа:
\begin{enumerate}
\item Анализ требований к эксперименту по поиску дейтронного ЭДМ методом определения частоты прецессии спина, обоснованных статистическими соображениями;
\item Изучение влияния неточности установки E+B спин-ротаторов на систематическую ошибку определения частоты ЭДМ прецессии (фальш-сигнал);
\item Изучение явления деполяризации пучка засчёт эффекта декогеренции спин-тюна, а также его подавление при помощи секступольных полей в неидеальной структуре;
\item Изучение влияния нарушения спин-прецессии на конечную статистику в частотном методе определения ЭДМ;
\item Моделирование процедуры калибровки ведущего магнитного поля кольца для его точного обращения при обратной инжекции пучка в процедуре CW/CCW.
\end{enumerate}

В перспективе предполагается использование предложенного метода поиска ЭДМ элементарной частицы в частотной области, а также разработанной в рамках данной работы методологии анализа колец, при конструировании кольца по поиску дейтронного ЭДМ на комплексе НИКА, Дубна.

\section{Актуальность и научная новизна работы}

В данной работе исследуется возможность измерения дейтронного ЭДМ в накопительном кольце, построенном на основе принципа Замороженного (или Квази-замороженного) спина, методом Frequency Domain.

Поиск ЭДМ в невырожденных системах был инициирован Эдвардом Пёрселлом и Норманом Рэмзи более 50 лет назад, для нейтрона. С тех пор было проведено множество всё более чувствительных экспериментов на нейтронах, атомах, и молекулах, и тем не менее, ЭДМ пока ещё не был обнаружен. На данный момент, верхний предел ЭДМ нейтрона оценивается на уровне $<3\cdot 10^{-26}$ e$\cdot$cm, протона --- $<8\cdot 10^{-25}$ e$\cdot$cm.~\cite{Pretz_presentation}

В 2008 году коллаборацией в Брукхейвенской Национальной Лаборатории (BNL, США) был предложен эксперимент по измерению ЭДМ дейтрона, основанный на использовании эффекта замороженного спина в магнитном накопительном кольце.~\cite{BNL} 

В 2015 году, коллаборацией Storage Ring EDM Collaboration, был предложен эксперимент по поиску протонного ЭДМ в полностью электрическом накопительном кольце.~\cite{Proton_EDM}

На данный момент, коллаборацией JEDI (Исследовательский центр ``Юлих,'' Германия) ведётся разработка структуры накопительного кольца для проведения предварительного эксперимента по измерению дейтронного ЭДМ на полностью магнитном синхротроне COSY. В рамках исследований, проводимых коллаборацией, была выдвинута концепция квази-замороженного кольца. Также, профессором Ю. Сеничевым была предложена методология поиска ЭДМ в частотной области (так называемый Frequency Domain Method).

Frequency Domain Method базируется на концепциях замороженного спина, предложенного в BNL, и Spin Wheel, предложенного профессором И. Коопом (Новосибирский Государственный Университет). Однако ни тот, ни другой метод сам по себе не решают проблему неточности установки элементов ускорителя, и возникающий в связи с этим фальш-сигнал.  На сегодняшний день, Frequency Domain --- это единственный известный, практически-реализуемый метод, решающий данную проблему.

\section{Структура работы}
Структура работы следует пунктам программы обозначенной в разделе~\ref{sec:research_program}

\chapter{Характеристика современного состояния изучаемой проблемы}
Предварительный библиографический список по теме диссертации.
Диссертация Еремея Валетова.~\citep[p.~235]{Eremey:Thesis}
Диссертация Андрея Иванова.
Диссертация Станислава Чекменева.
AGS Proposal (deuteron).
Senichev: Investigation of lattice for deuteron EDM ring.
Senichev: Spin tune decoherence effects in electro- and magnetostatic structures.
Senichev: Search for the charged particle electric dipole moments in storage rings.
Senichev: Frequency domain method of the search for the deuteron electric dipole moment in a storage ring with imperfections.
JEDI: Spin Tune mapping.


\chapter{Результаты разработки основных направлений теоретической концепции (FS и QFS колец)}
\section{Статистический анализ требований к эксперименту по поиску
  ЭДМ}

Вводные рассуждения по поводу измерения поляризации, проблемы
деполяризации пучка, и т.п.

\subsection{Модель частоты событий на поляриметре}
В наших рассуждениях мы предположили следующую простую модель
измеряемого сигнала:
\begin{equation}\label{eq:DetCntRt}
	N(t) = N_0(t)\cdot\bkt{1 + P\cdot e^{-\sfrac{t}{\LTd}}\cdot\sin(\omega\cdot t + \phi)},
\end{equation}
где $N_0(t)$ частота событий, связанная с неполяризованным сечением,
$\LTd$ время жизни поляризации, связанное с декогеренцией.

Ток пучка, рассеиваемого на мишени может быть описан с помощью:
\[
I(t)= I_0\cdot e^{t/\LTb} = \nu N_0^b\cdot e^{t/\LTb},
\]
где $\LTb$ --- время жизни пучка, $N_0^b$ его начальное число частиц,
и $\nu$ частота оборота пучка в ускорителе. Обозначая $p$ вероятность
что рассеянная частица полетит в сторону детектора, ожидаемое число
частиц, детектируемых в течении времени измерения $\dtc$ может быть
записано как
\begin{align}
N_0(t) & = p\cdot\int_{-\dtc/2}^{+\dtc/2} I(t+\tau)\td\tau \notag                    \\
& = p\cdot\frac{\nu N_0^b}{\lamb} e^{\lamb t}\cdot \bkt{e^{\lamb\sfrac{\dtc}{2}} - e^{-\lamb\sfrac{\dtc}{2}}} \notag \\
& \approx \underbrace{p\cdot\nu N_0^b e^{\lamb t}}_{\text{rate}~r(t)} \cdot\dtc.
\end{align}
Таким образом, получаем распределение Пуассона
\[
	P_{N_0(t)}(\tilde{N}_0) = \frac{\bkt{r(t)\dtc}^{\tilde{N}_0}}{\tilde{N}_0!}\cdot e^{-r(t)\dtc},
\]
с дисперсией $\SD{\tilde{N}_0}^2(t) = N_0(t)$. %In the limit of large $N_0(t)$, one can use the Gaussian approximation.

Нас интересует ожидаине $N_0(t) = \Xpct{\tilde{N}_0(t)}$, и его
стандартное отклонение $\SD{N_0}(t)$. Обозначая время измерения одного
события $\dtm$, полное время измерений $\dtc$, и число событий за
измерение $\Ncm = \dtm/\dtc$, ожидание
\begin{equation*}
	\Xpct{\tilde{N}_0(t)}_{\dtm} &= \frac{1}{\Ncm}\sum_{i=1}^\Ncm \tilde{N}_0(t_i).
\end{equation*}
Поскольку это сумма случайных переменных, $N_0(t)$ имеет нормальное
распределение; тогда стандартное отклонение среднего % abuse of notation here (SD in place of SE) for aesthetic reasons
\begin{align*}
\SD{N_0}(t) & = \SD{\tilde{N}_0}(t)/\sqrt{\Ncm} = \sqrt{N_0(t)\frac{\dtc}{\dtm}}            \\
& \approx \sqrt{\frac{p\cdot\nu N_0^b}{\dtm}}\cdot\dtc \cdot\exp\bkt{\frac{\lamb}{2}\cdot t}.
\end{align*}
\newcommand{\A}{\frac{1}{\sqrt{p\cdot\nu N_0^b}}}

Отметим, что относительная ошибка растёт со временем:
\begin{equation}\label{eq:MeasRelErr}
	\frac{\SD{N_0}(t)}{N_0(t)} \approx \frac{A}{\sqrt{\dtm}}\cdot\exp\bkt{-\frac{\lamb}{2}t} = \frac{A}{\sqrt{\dtm}}\cdot\exp\bkt{\frac{t}{2\LTb}},~ A=\A.
\end{equation}

\subsection{Асимметрия сечения}
\newcommand{\Asym}{\mathcal{A}}
В качестве меры поляризации пучка используют асимметрию частоты
событий детекторов.~\citep[p.~17]{Eversmann:Thesis} Асимметрия сечения
взаимодействия --- это нормализованная разность числа событий (в
единицу времени) на
детекторах, расположенных по разные стороны от вакуумной камеры:
\begin{equation}\label{eq:AsymDef}
	\Asym = \frac{N(\frac\pi2) - N(-\frac\pi2)}{N(\frac\pi2)+N(-\frac\pi2)}.
\end{equation}

%% In the simulation to follow, the function fitted to the asymmetry data is
%% \begin{equation}\label{eq:xFOM}
%% 	\Asym(t) = \Asym(0)\cdot e^{\lamd\cdot t}\cdot\sin\bkt{\omega\cdot t + \phi},
%% \end{equation}
%% with three nuisance parameters $\Asym(0)$, $\lamd$, and $\phi$. 

В связи с уменьшением числа частиц в пучке, измерение асимметрии
сечения гетероскедастично. Из~\citep[p.~18]{Eversmann}, мы приняли
модель гетероскедастичности
\begin{equation}\label{eq:AsymHtsk}
	\SD{\Asym}^2(t) \approx \frac{1}{2N_0(t)}.
\end{equation}

\subsection{Временное окно измерений}
\DeclareDocumentCommand{\stat}{s}{\IfBooleanTF{#1}{X_{tot}}{\frac{\SD{\meas}^2}{\SE{\hat\omega}^2\cdot \var[w]{t}}}}
\DeclareDocumentCommand{\var}{O{}mo}{\mathrm{var}_{#1}\bkt*{#2\IfValueT{#3}{\vert~ #3}}}
\newcommand{\dtnd}{\dt_{zc}}
\newcommand{\SNR}{\text{SNR}}

Предполагая нормальное распределение ошибки измерений, с нулевым
ожиданием и дисперсией $\SD{\meas}^2$, эстиматор максимального
правдоподобия дисперсии оценки частоты колебаний асимметрии сечения
взаимодействия $\Asym$ может быть выражен как
\begin{align}
\var{\hat\omega} &= \frac{\SD{\meas}^2}{X_{tot}\cdot \var[w]{t}}, \label{eq:VarW}
\shortintertext{где}
X_{tot} &= \sum_{j=1}^{\Nm} x_j = \sum_{s=1}^{\Nnd}\sum_{j=1}^{\Nmnd} x_{js}, \notag\\
\var[w]{t} &= \sum_i w_i \bkt{t_i - \avg{t}[w]}^2,~ \avg{t}[w] = \sum_i w_i t_i, \notag\\
w_i &= \frac{x_i}{\sum_j x_j},~ x_i = (\Asym(0)\exp(\lamd t_i))^2\cos^2(\omega t_i + \phi) = \bkt{\mupp}^2. \notag
\end{align}

В выражении выше, $X_{tot}$ есть полная информация Фишера сэмпла, и
$\var[w]{t}$ --- мера длительности его измерения. Можно наблюдать, что
выбирая подходищие моменты времени для измерения, можно увеличить
фактор $X_{tot}$, поскольку он пропорционален сумме временных
производных сигнала. Если частота и фаза колебаний уже известны до
приемлемого уровня, можно дальше улучшить эффективность измерений,
применяя схему измерений в которой выбираются только моменты быстрого
изменения сигнала (модуляция сэмплинга).

Оба фактора $\var[w]{t}$ и $X_{tot}$ ограничены конечным временем
жизни поляризации. Можно выразить $\sum_{j=1}^{\Nmnd} x_{js} = \Nmnd
\cdot x_{0s}$, для некоторого среднего значения $x_{0s}$ в данном узле
$s$, где $\Nmnd$ измерений асимметрии на узле. Будем называть
\emph{временем сжатия} (обозначение $\dtnd$) период времени вокруг узла
сигнала, в течении которого производятся измерения. Значение суммы
$\sum_{j=1}^{\Nmnd} x_{js}$ спадает экспоненциально из-за
деполяризации, так что $x_{0s} = x_{01}\exp{(\lamd\cdot \frac{(s-1)\cdot\pi}{\omega})}$. Тогда:
\begin{align}
	X_{tot} & = \Nmnd\cdot x_{01} \cdot \frac{\exp{\bkt{\frac{\lamd\pi}{\omega}\Nnd}}-1}{\exp{\bkt{\frac{\lamd\pi}{\omega}}}-1} 
	\equiv \Nmnd \cdot x_{01}\cdot g(\Nnd); \label{eq:FItot}\\
	x_{01}  & = \frac{1}{\dtnd}\int_{-\dtnd/2}^{+\dtnd/2}\cos^2(\omega\cdot t)\td t = \frac12\cdot \bkt{1 + \frac{\sin\omega\dtnd}{\omega\dtnd}},                                    \label{eq:MeanFIZC}   \\
	\Nmnd   & = \frac{\dtnd}{\dtm}. \label{eq:NumMeasNode}
\end{align}

Уравнение~\eqref{eq:FItot} может быть использовао чтобы оценить
пределы длительности эксперимента. В Таблице~\ref{tbl:FItot} мы
собрали: процент от предела информации
Фишера сэмпла, время (как фактор времени жизни поляризации) исчерпания этого
процента информации из сэмпла, и соответствующее этому времени
отношение сигнал-шум. Отношение сигнал-шум вычислено по формуле:
\begin{equation}\label{eq:TauRatioSNR}
  \SNR = \frac{\Asym(0)\cdot e^{-\sfrac{t}{\LTd}}}{\SD{\Asym}(t)} 
  \approx \sqrt{2\cdot p\cdot\nu N_0^b\cdot \dtc}\cdot \Asym(0)\cdot \exp\bkt*{-\frac{t}{\LTd}\cdot\bkt{1+\frac12\frac{\LTd}{\LTb}}},
  %	 \notag \\
  %	&\approx \Asym(0) \exp\bkt*{-\frac{t}{\LTd}\cdot\bkt{1+\frac12\frac{\LTd}{\LTb}}},
\end{equation}
%in which the factor before $\Asym(0)$ is approximately equal to 1 as a result of: $\SD{N_0}(0)/N_0(0)\approx 3\%$ with 2000 polarimetry measurements per asymmetry measurement ($\dtm = 2000\cdot\dtc$).
в которой, полагая $\SD{\Asym(0)}/\Asym(0) \approx 3\%$ (точность
измерения поляриметрии), коэффициент перед экспонентой равен 33.
\begin{table}[h]
  \centering
  \caption{Выбранная информация Фишера, длительность измерений,
    соответствующее отношение сигнал-шум.\label{tbl:FItot}}
  \begin{tabular}{rrr}
    \hline
    Предел ИФ(\%) & длительность ($\times\LTd$) &  SNR \\ \hline
    95 &                    3.0 &  0.4 \\
    90 &                    2.3 &  1.1 \\
    70 &                    1.2 &  5.5 \\
    50 &                    0.7 & 11.7 \\ \hline
  \end{tabular}
\end{table}

Предполагая отсутствие деполяризации ($\lamd=0$) и однородный сэмплинг
с частотой $1/\dt$, уравнение~\eqref{eq:VarW} может быть записано
через физические переменные как
\begin{align*}
  \stat* &= \sum_{k=1}^K \Asym^2(0)\cos^2(\omega t_k + \phi) = \frac12 \Asym^2(0)\cdot K, \\
	\var[w]{t} &= \sum_{k=1}^K(k\dt - \avg{t}[w])^2\underbrace{w_k}_{1/K} \\
				&\approx \frac{\dt^2}{12}K^2 = \frac{T^2}{12},
\intertext{и тогда}					
	\var{\hat{\omega}} &= \frac{24}{KT^2}\cdot\bkt{\frac{\SD{\meas}}{\Asym(0)}}^2.
\end{align*}


\section{Систематическая ошибка в эксперименте, связанная с
  неточностью установки магнитов}\label{sec:SystErr:FakeSignalSim}

Систематические ошибки, вызванные физическими неидеальностями
ускорителя, включая неточность юстировки оптических элементов,
вызывают фальш-сигнал ЭДМ.~\citep[p.~230]{Eremey:Thesis} Особенно в
этом отношении проблематичны наклоны элементов вокруг оптической оси, поскльку они
индуцируют паразитные горизонтальные компоненты магнитного поля $B_x$
и $B_z$, которые обе вращают спин в вертикальной плоскости; той, в которой измеряется ЭДМ.

Ю. Сеничевым были сделаны~\cite{Senichev:FDM} аналитические оценки МДМ частоты прецессии спина
вокруг радиальной оси. Из уравнения Т-БМТ, и выражения силы Лоренца,
скорость МДМ прецессии вокруг радиальной оси есть
\begin{equation}
  \SD{\W_x^{MDM}} = \frac{q}{m\gamma}\frac{G+1}{\gamma}\frac{\SD{B_x}}{\sqrt{n}},
\end{equation}
где $n$ есть число наклонённых спин-ротаторов, и $\SD{B_x} = B_y
\SD{\delta h}/L$, при стандартном отклонении ошибки юстировки
$\SD{\delta h}$. При величине ошибки $\SD{\delta h} = 100$ мкм, и
длине дефлектора $L=1$ м, $\SD{\W_x^{MDM}} \approx 100$ рад/сек.~\cite{Senichev:FDM}

Мы изучили спиновую динамику в структурах с замороженным и
квази-замороженным спином в присутствии наклонов оптических элементов
с помощью кода COSY INFINITY. Наши симуляции согласуются с оценками,
представленными выше.

\subsection{Имплементация паразитного поля}\label{sec:ErrorFieldImplementation}
Имплементирую неидеальности полей, мы следовали рекомендациям
изложенным в~\citep[p.~235]{Eremey:Thesis}. Малое возмущение
магнитного поля, в первом приближении, действует как маленький пропорциональный поворот
спин-вектора. Поэтому мы имплементировали наклон E+B элемента как
домножение соответствующей матрицы поворота на его спиновую матрицу
перехода, ``спин-кик.''

В соответствии с уравнением~\eqref{eq:TMBT_MDM}, изменение МДМ частоты
прецессии, ассоциированное с введённым паразитным полем $(B_x, 0, B_z)$ есть
\begin{align*}
  \Delta\W_{MDM} &= \frac qm (B_x, 0, B_z),
  \intertext{поэтому угол спин-кика равен}
  \Theta_{kick} &= t_0\Delta\W_{MDM},
\end{align*}
где $t_0 = L/v_0$ пролётное время референсной частицы через элемент.

\section{Декогеренция спина в идеальном накопительном кольце}
Когеренцией спина называется мера или качество сохранения поляризации
в изначально поляризованном пучке.~\citep[стр.~205]{Eremey:Thesis}

Когда поляризованный пучок инжектируется в накопительное кольцо, спин
векторы частиц пучка начинают прецессировать вокруг вертикального
(Ведущего) поля. частота прецессии зависит от равновесного уровня
энергии частицы, который различен для частиц пучка.

Это обстоятельство не является проблемой в том случае, когда начальная
поляризация пучка вертикальна; однако метод измерения ЭДМ в
накопительном кольце, основанный
на принципе замороженного спина требует, чтобы вектор поляризации
пучка был сонаправлен с его вектором импульса, т.е. лежал в
горизонтальной плоскости. Таким образом, декогеренция спина есть
внутренняя проблема метода замороженного спина.
\subsection{Требования ко времени когеренции пучка}
Время когеренции спина (spin coherence time; SCT) для метода
замороженного спина, выполненного в накопительном кольце с идеально
установленными элементами определяется минимальным детектируемым углом
отклонения вектора поляризации пучка из горизонтальной плоскости
только засчёт ЭДМ. Для уровня чувствительности $10^{-29}~e\cdot cm$
это примерно $5\cdot10^{-6}$.~\cite{BNL:Deuteron2008}

В соответствии с уравнением Т-БМТ,
\[
\W_{EDM,x} = \eta\frac{qE_x}{2mc},
\]
где $\eta$ есть коэффициент пропорциональности между ЭДМ и спином,
равный $10^{-15}$ для дейтрона,для данного уровня чувствительности.~\citep[p.~206]{Eremey:Thesis}

Для дейтронного BNL FS кольца, $E_x = 12$
МВ/м,~\citep[p.~19]{BNL:Deuteron2008} так что $\W_{EDM,x}\approx
10^{-9}$ рад/сек. Таким образом получаем, что для того, чтобы достичь
детектируемый уровень отклонения вектора поляризации на 1 мкрад требуется SCT порядка 1000 секунд.~\citep[p.~207]{Eremey:Thesis}
\subsection{Происхождение декогеренции}
Декогеренция спина в пучке вызвана разницей угловых скоростей
прецессии спинов частиц, которая, в свою очередь, вызвана разницей
длин орбит и импульсов частиц. Это можно видеть исходя из следующих
соображений.

Когда частица со спином входит в область магнитного поля, её спин-вектор
начинает поворачиваться вокруг вектора магнитного поля с угловой
скоростью определяемой уравнением Т-БМТ~\eqref{eq:TBMT_MDM}:
\begin{equation*}
  \vec\W_{MDM} = \frac qm G\vec B.
\end{equation*}
На выходе из области, вектор спина повёрнут на угол
\begin{equation*}
  \theta = \Delta t\cdot\W_{MDM} = \frac Lv \cdot \frac qm GB\cdot \frac{\gamma_0}{\gamma_0} = \frac{L\gamma_0 GB}{B\rho} = \frac L\rho\gamma_0\cdot G,
\end{equation*}
где $L$ есть длина пути внутри области с магнитным полем, и $B\rho =
p/q$ магнитная жёсткость.

В простой модели рассмотренной выше, влияние орбитальной динамики на
спиновую динамику вырадено через $\gamma_0 L/\rho$ (эффективный
Лоренц-фактор). В случае референсной частицы, $\gamma_0L/\rho =
\gamma_0\alpha$, $\alpha$ угол поворота вектора импульса,в то время
как для частицы участвующей в бетатронном движении, эффективный
Лоренц-фактор больше. В следующих разделах мы выразим связь между
спиновой и орбитальной динамиками частицы в накопительном кольце в
более общих терминах.

\subsubsection{Сдвиг равновесного значения импульса частицы}
Продольная динамика заряженной частицы на референсной орбите в
накопительном кольце описывается системой уравнений:
\begin{equation*}
  \begin{cases}
    \ddt{\varphi} &= -\w_{RF}\eta\delta, \\
    \ddt{\delta} &= \frac{q V_{RF}\w_{RF}}{2\pi h\beta^2E}\sin\varphi.
  \end{cases}
\end{equation*}
В уравнениях выше: $\varphi$ отклонение фазы частицы от референсной
$\varphi_0 = 0$; $\delta = \frac{\Delta p}{p_0}$ относительное
отклонение импульса от $p_0$ референсной частицы; $V_{RF}$, $\w_{RF}$
амплитуда и частоты колебаний ВЧ поля; $\eta = \alpha_0 - \gamma^{-2}$
слип-фактор, где$\alpha_0$ --- коэффициент сжатия орбиты, определяемый
чяерез$\sfrac{\Delta L}{L} = \alpha_0\delta$, $L$ длина орбиты; $h$
гармоническое число; $E$ полная энергия ускоряемой частицы. $\w_{RF} =
2\pi h f_{rev}$, где $f_{rev}=T_{rev}^{-1}$ --- частота оборота пучка.

Решения этой системы формируют семейство эллипсов в плоскости
$(\varphi, \delta)$, центрированных на $(0,0)$. Однако, если
рассмотреть частицу, участвующую в бетатронных колебаниях, и
использовать разложение Тейлора более высокого порядка для
коэффициента сжатия орбиты $\alpha = \alpha_0 + \alpha_1\delta$,
первое уравнение системы превратится в:~\citep[p.~2579]{Senichev:IPAC13}
\[
\ddt{\varphi} = -\w_{RF} \bkt*{\bkt{\frac{\Delta L}{L}}_\beta + \bkt{\alpha_0 + \gamma^{-2}}\delta + \bkt{\alpha_1 - \alpha_0\gamma^{-2} + \gamma^{-4}}\delta^2},
\]
где $\bkt{\frac{\Delta L}{L}}_\beta =
\frac{\pi}{2L}\bkt*{\varepsilon_xQ_x + \varepsilon_yQ_y},$ есть
удлинение орбиты, связанное с бетатронным движением; $\varepsilon_x$ и
$\varepsilon_y$ --- горизонтальный и вертикальный эмиттансы пучка, и
$Q_x$ и $Q_y$ горизонтальный и вертикальный тюны.~\citep[p.~2580]{Senichev:IPAC13}

Решения модифицированной системы более не центрированы на одной и той
же точке. Удлинение орбиты и отклонение импульса вызывают сдвиг
равновесного значения импульса частицы~\citep[p.~2581]{Senichev:IPAC13}
\begin{equation}\label{eq:EquLevMom_shift}
  \Delta\delta_{eq} = \frac{\gamma_0^2}{\gamma_0^2\alpha_0 - 1}\bkt*{\frac{\delta_m^2}{2}\bkt{\alpha_1 - \alpha_0\gamma^{-2} + \gamma_0^{-4}} + \bkt{\frac{\Delta L}{L}}_\beta},
\end{equation}
где $\delta_m$ --- амплитуда синхротронных колебаний.

\subsubsection{Понятие эффективного Лоренц-фактора}
Равновесное значение энергии, связанное со сдвигом импульса~\eqref{eq:EquLevMom_shift}, называнное \emph{эффективным Лоренц-фактором}, есть~\cite{Senichev:FDM}
\begin{equation}\label{eq:EffectiveGamma}
  \gamma_{eff} = \gamma_0 + \beta_0^2\gamma_0\cdot\Delta\delta_{eq},
\end{equation}
где $\gamma_0$, $\beta_0$ --- Лоренц-фактор референсной частицы и
нормализованное значение
скорости. Уравнения~\eqref{eq:EquLevMom_shift}
и~\eqref{eq:EffectiveGamma} определяют связь между спиновой и
орбитальной динамиками частицы.


\subsection{Подавление декогеренции с помощью секступольных полей}
Чтобы минимизировать декогеренцию спина, связанную с бетатронным
движением и отклонением импульса, могут быть использованы
секступольные (или октупольные) поля~\citep[p.~212]{Eremey:Thesis}

Секступоль силы
\[
S_{sext} = \frac{1}{B\rho} \pddx{B_y}[x][2],
\]
где $B\rho$ магнитная жёсткость, влияет на коэффициент сжатия орбиты
первого порядка как~\citep[p.~2581]{Senichev:IPAC13}
\begin{align}
  \Delta \alpha_{1,sext} &= -\frac{S_{sext}D_0^3}{L}, \label{eq:Sext_compaction_effect}
  \intertext{и одновременно на длину орбиты как}
  \bkt{\frac{\Delta L}{L}}_{sext} &= \mp \frac{S_{sext}D_0\beta_{x,y}W_{x,y}}{L}, \label{eq:Sext_OL_effect}
\end{align}
где $D(s,\delta) = D_0(s) + D_1(s)\delta$ обозначает функцию дисперсии.

В следующих разделах мы будем называть декогеренцию, связанную с
горизонтальными/вертикальными бетатронными, и синхротронными
колебаниями соответственно X-/Y-, и D-декогеренцией. 

Из уравнений~\cref{eq:Sext_compaction_effect,eq:Sext_OL_effect} можно
видеть, что для подавления декогеренции необходимы три семейства
секступолей, помещённых в максимумы функций: $\beta_x$, $beta_y$ для подавления
X-,Y-декогеренции, и $D_0$ для D-декогеренции.

\section{Смена полярности ведущего магнитного поля}
большая проблема.

\chapter{Результаты численного моделирования спин-орбитальной динамики
  частиц в FS (и QFS) кольцах}

\section{Равномерная и модулированная схемы измерения поляризации}
Мы симулировали сбор данных двух детекторов с параметрами собранными в
Таблице~\ref{tbl:DetCntRtParam} на протяжении $T_{tot}=1000$ секунд,
выбираемыми равномерно по времени с частотой $f_s = 375$ Гц.

Данные параметры симуляции были выбраны исходя из следующих
рассуждений: число частиц в пучке порядка $10^{11}$; если мы хотим
сохранить время жизни пучка равным времени жизни поляризации, мы не
можем исчерпать более 75\% от его начального числа частиц; всего лишь
1\% всех рассеяний на мишени полезны для поляриметрии, так что
остаётся $\vp{7.5}{8}$ полезных рассеяний. Измерение частоты событий
$N_0(t)$ с точностью примерно 3\% требует приблизительно 2000 событий
на детекторе, что ещё уменьшает число измерений до $\vp{3.75}{5}=
f_s\cdot T_{tot}$. Ожидаемая длительность цикла 1000 секунд, отсюда $f_s = 375$ Гц. 

Относительная ошибка измерения скоростей счёта детекторов отражена на
Рисунке~\ref{fig:LRDetErr}; асимметрия сечения, вычисленная в
соответствии с уравнением~\eqref{eq:AsymDef}, представлена на Рисунке~\ref{fig:Asym}.
Данные асимметрии фитируются нелинейной, гетероскедастичной моделью
заданной как
\[
\Asym(t) = \Asym(0)\cdot e^{\lamd\cdot t}\cdot\sin\bkt{\omega\cdot t + \phi},
\]
с функцией дисперсии весов заданной
уравнением~\eqref{eq:AsymHtsk}. Результаты фитирования представлены в Таблице~\ref{tbl:FitRes}.
\begin{table}[h]
  \begin{minipage}[t]{.5\linewidth}
    \centering
    \caption{Параметры модели частоты событий детекторов\label{tbl:DetCntRtParam}}
    \begin{tabular}[t]{cccc}
      \hline
      &   Левый   &     Правый     &  \\ \hline
      $\phi$  & $-\pi/2$ &   $+\pi/2$    &   рад   \\
      $\omega$ &  \multicolumn{2}{c}{3}   & рад/сек \\
      $P$    & \multicolumn{2}{c}{0.4}  &  \\
      $\LTd$  & \multicolumn{2}{c}{721}  &   сек   \\
      $\LTb$  & \multicolumn{2}{c}{721}  &   сек   \\
      $N_0(0)$ & \multicolumn{2}{c}{6730} &  \\ \hline
    \end{tabular}
  \end{minipage}%
  \begin{minipage}[t]{.5\linewidth}
    \centering
    \caption{Результаты фитирования\label{tbl:FitRes}}
    \begin{tabular}[t]{crrc}
      \hline
      & Оценка &             Ст. Ошибка &  Единицы   \\ \hline
      $\Asym(0)$ &   0.400 & $\vp{9.03}{-5}$ &         \\
      $\lamd$   &  -0.001 & $\vp{7.86}{-7}$ &  1/сек  \\
      $\omega$  &   3.000 & $\vp{7.55}{-7}$ & рад/сек \\
      $\phi$   &  -1.571 & $\vp{2.25}{-2}$ &   рад   \\ \hline
    \end{tabular}
  \end{minipage}
\end{table}

\begin{figure}[H]
	\centering
	\includegraphics[width=\textwidth]{edm_img/LR_detector_relErr}
	\caption{Relative count rate measurement error for the left and right detectors as a function of time.\label{fig:LRDetErr}}
\end{figure}

\begin{figure}[H]
	\centering
	\includegraphics[width=\textwidth]{edm_img/Asymmetry}
	\caption{Expectation value (black line) and sample measurements (red dots) of the cross-section asymmetry.\label{fig:Asym}}
\end{figure}

Если начальная оценка частоты, полученная из равномерно собранного
сэмпла, имеет стандартную ошибку порядка $10^{-6}$ рад/сек, симуляции
подтверждают, что стандартная ошибка оценки может быть улучшена до примерно $\vp{5.8}{-7}$ рад/сек.

\section{Фальш-сигнал, связанный с неточностью установки магнитов}
Данная серия симуляций была проведена с целью подтвердить два тезиса
касательно систематической ошибки измерения частоты прецессии спина в
вертикальной плоскости, вызванной неточностью установки E+B элементов:
\begin{inparaenum}[1)]
\item индуцированный МДМ-эффект зависит только от среднего значения
  угла наклона элементов, но не от  конкретной последовательности
  углов (т.е. отсутствует эффект \emph{геометрической фазы}); и
\item эта зависимость носит линейный характер.
\end{inparaenum}

Наклон элемента вокруг оптической оси моделировался путём добавления
после элемента спин-кика вокруг радиальной оси соответствующей
величины (см. раздел~\ref{sec:ErrorFieldImplementation}). Это
гарантирует сохранение замкнутой орбиты при введении наклонов, что
физически обусловлено появлением компенсирующего электрического поля 
спин-ротатора при его наклоне.

Симуляция была проведена следующим образом: мы распределили наклоны
$\Theta_{tilt}$ E+B элементам FS структуры случайным образом. После
построения матриц перехода 3-го порядка, были вычислены разложения
Тейлора функций спин-тюна и оси прецессии спина (SPA). Члены нулевого
порядка этих разложений представляют собой спин-тюн и SPA референсной частицы.

Симуляция была проведена 11 раз; каждый раз углы наклона
спин-ротаторов выбирались из нормального распределения
$N(\mu_0\cdot(i-5), \sigma_0)$, где $\mu_0 = 10\cdot \sigma_0 =
10^{-4}$ рад, $i\in\lbrace0,\dots, 10\rbrace$. Результаты представлены
на Рисунке~\ref{fig:Linearity_test_shifting_gauss}. На
Рисунке~\ref{fig:Linearity_test_compensated} показаны результаты,
когда три пары E+B повёрнуты на противоположные углы, повёрнут на угол
$\mu_i = (i-5)\cdot 10^{-6}$ рад,
$i\in\lbrace0,\dots,10\rbrace$. Симуляции были выполнены на энергии
270.0092 МэВ.\footnote{На этой энергии ось прецесии спина и спин-тюн
  не определены в системе координат связанной с пучком, использованной
  COSY INFINITY, для идеальной структуры. Это соответствует ситуации
  когда спин не прецессирует ни в какой плоскости (горизонтальной или
  вертикальной), что есть условие замороженного спина в идеальной структуре.}

\begin{figure}[H]
  \includegraphics[width=\textwidth]{edm_img/linearity_test_shifting_gauss}
  \caption{Ось прецессии спина и частоты прецессии для неидеальной FS
    структуры, при наклонах E+B элементов.\label{fig:Linearity_test_shifting_gauss}}
\end{figure}
\begin{figure}[H]
  \includegraphics[width=\textwidth]{edm_img/linearity_test_compensated+microrad}
  \caption{Три пары противоположно наклонённых E+B элементов, плюс
    нескомпенсированный элемент.\label{fig:Linearity_test_compensated}}
\end{figure}


\section{Декогеренция}
При проведении нижеследующих тестов симулировалась инжекция
плоского, гауссовского пучка в структуру с замороженным
спином. Инжектируемые пучки состояли из 30 частиц, распределённых в
плоскости $y-z$ как $y\sim N(y_0, 10^{-4})$ [м]; $x,d =
0$. Оффсет $y_0$ варьировался в диапазоне $[-1, +1]$ мм. Начальное
направление спин-векторов всех частиц --- продольное: $\vec S(t=0) = (0,0,1)$.

Также в структуре варьировалось значение градиента GSY секступоля,
модулирующего декогеренцию в вертикальной плоскости. GSY менялся в
диапазоне $[GSY0 - 5\cdot10^{-3}, GSY0 + 5\cdot10^{-3}]$, где
$GSY0=-2.5e-3$ --- оптимальное значение градиента для идеальной структуры.

На каждое значение градиента приходится 10 инжекций.

Пучок инжектировался на энергии 270.0092 МэВ (строгий FS), в структуру
с неточно-установленными E+B спин-ротаторами. 

Наклоны E+B элементов
генерировались из распределения $N(0, 5\cdot10^{-4})$ радиан. При
симуляциях использовалась энергия строгой заморозки спина, а не,
например, близкая к ней 270 МэВ, для того, чтобы минимизировать
вертикальную компоненту оси прецессии. Матрицы перехода орбитального и спинового движений строятся до  третьего
порядка разложения ряда Тейлора, чтобы обеспечить устойчивость
процедуры TSS COSY Infinity.~\cite{COSYINF:BeamPhysMan}

Далее ансамбль начальных значений, представляющий пучок, трекается
через структуру на протяжении $1.2\cdot10^6$ оборотов, что
примерно эквивалентно 1.2 секундам. Каждые 800 оборотов производится
запись необходимых для анализа данных.

Собираемые данные: 
\begin{inparaenum}[\itshape a\upshape)]
\item результаты вычислений процедуры TSS: спин-тюн $\nu_s$ и компоненты вектора оси инвариантного спина
  $\bar n$, а также
\item компоненты спина $(S_X, S_Y, S_Z)$, и фазового пространства $(X,A,Y,B,T,D)$.
\end{inparaenum}

Из данных по компонентам спина вычисляется поляризация банча по
формуле
\[
\vec P = \frac{\sum_i\vec s_i}{|\sum_i\vec s_i|}.
\]

Поляризация фиритуется функцией $f(t; a,f,\phi) = a\cdot \sin(2\pi\cdot
f\cdot t + \phi)$, оцениваются все три параметра $(\hat a, \hat f,
\hat\phi)$. 

\subsection{Симуляция эффекта подавления декогеренции спина в вертикальной плоскости при помощи секступолей}
\begin{figure}[H]
  \centering
  \begin{subfigure}[b]{\textwidth}
    \includegraphics[width=\linewidth]{\multisext/ny_vs_offset}
    \caption{Вертикальная компонента оси прецессии спина $\bar n_y$ в зависимости
      от вертикального смещения центра пучка.\label{fig:DECOH_full_ny}}
  \end{subfigure}
\end{figure}
\begin{figure}[H]\ContinuedFloat
  \begin{subfigure}[b]{\textwidth}
    \includegraphics[width=\linewidth]{\multisext/ny_vs_offset_zoom}
    \caption{Деталировка Рисунка~\ref{fig:DECOH_full_ny}. Вертикальная компонента $\bar
      n_y$ (и $\bar n_x$) параболична вокруг референсной орбиты при
      оптимальном значении градиента GSY Y-секступоля, в отличии от
      $nu_s$, который линеен.}
  \end{subfigure}
\end{figure}
\begin{figure}[H]\ContinuedFloat
  \centering
  \begin{subfigure}[b]{\textwidth}
    \includegraphics[width=\linewidth]{\multisext/spin_tune_vs_offset}
    \caption{Спин-тюн $\nu_s$.}
  \end{subfigure}
  \caption{Данные DECOH построенные для каждого значения градиента GSY
    в зависимости от вертикального оффсета пучка.}
\end{figure}

\begin{figure}[H]
  \centering
  \begin{subfigure}[b]{\textwidth}
    \includegraphics[width=\linewidth]{\multisext/FreqY_vs_offset}
    \caption{Полный диапазон.\label{fig:FreqY_vs_offset}}
  \end{subfigure}
\end{figure}
\begin{figure}[H]\ContinuedFloat
  \begin{subfigure}[b]{\textwidth}
    \includegraphics[width=\linewidth]{\multisext/FreqY_vs_offset_zoom}
    \caption{Деталировка Рисунка~\ref{fig:FreqY_vs_offset}. Оценка
      частоты колебаний вертикальной компоненты поляризации зависит от
      начального оффсета пучка линейно, как спин-тюн $\nu_s$, а не как $\bar n_y$.}
  \end{subfigure}
  \caption{Оценка частоты прецессии поляризации пучка в вертикальной
    плоскости в зависимости от начального оффсета пучка от референсной
    орбиты для оптимального значения градиента GSY секступоля
    (оранжевый), и двух значений на противоположных концах
    рассматриваемого спектра значений GSY.}
\end{figure}

\subsection{Исследование зависимости оценки частоты прецессии поляризации банча от спин тюна и прецессии оси стабильного спина}

\begin{figure}[H]
  \centering
  \includegraphics[width=\linewidth]{\decoh/ny_vs_turn}
  \caption{Вертикальная компонента $\bar n$ для частиц с оффсетами,
    соответственно.: [1.02749, 1.02937, 1.02840] мм. Мы наблюдаем
    быстрые осцилляции вокруг некоторого среднего уровня. Этот средний
    уровень изменяется параболически с вертикальным оффсетом частиц
    (см. Рисунок~\ref{fig:mean_tune_axis} ниже). Быстрые осцилляции
    вызваны бетатронным движением (см. Рисунки~\ref{fig:tune_axis_position_y},~\ref{fig:tune_axis_position_x}).\label{fig:ny_vs_turn}}
\end{figure}

\begin{figure}[H]
  \centering
  \begin{subfigure}[b]{\textwidth}
    \includegraphics[width=\linewidth]{\decoh/ny_vs_y}
    \caption{Вертикальная компонента $\bar n$ в зависимости от
      вртикального положения.}
  \end{subfigure}
\end{figure}
\begin{figure}[H]\ContinuedFloat
  \begin{subfigure}[b]{\textwidth}
    \includegraphics[width=\linewidth]{\decoh/spin_tune_vs_y}
    \caption{Спин-тюн в зависимости от вертикального положения.}
  \end{subfigure}
  \caption{Частота прецессии частицы в зависимости от её вертикального
    оффсета. Выраженная нефункциональность зависимости
    параметров от вертикального положения частицы, отражённая на
    рисунках --- следствие их зависимости также и от радиального положения
    частицы, которое также осциллирует на малой амплитуде (см. Рисунок~\ref{fig:tune_axis_position_x}). \label{fig:tune_axis_position_y}}
\end{figure}

\begin{figure}[H]
  \centering
  \begin{subfigure}[b]{\textwidth}
    \includegraphics[width=\linewidth]{\decoh/ny_vs_x}
    \caption{Вертикальная компонента $\bar n$ в зависимости от
      радиального положения.}
  \end{subfigure}

  \begin{subfigure}[b]{\textwidth}
    \includegraphics[width=\linewidth]{\decoh/spin_tune_vs_x}
    \caption{Спин-тюн в зависимости от радиального положения.}
  \end{subfigure}
  \caption{Частота прецессии спина в зависимости от радиального
    положения частицы.\label{fig:tune_axis_position_x}}
\end{figure}


\section{Калибровка величины ведущего магнитного поля с помощью
  частоты прецессии поляризации пучка в горизонтальной плоскости}
Симуляции и анализ данных по этому поводу ещё ведутся.

\chapter{Результаты обобщения и систематизации результатов проведённых исследований}
\section{Связь между спин тюном и осью стабильного спина}
Концепция эффективного Лоренц-фактора предполагает равенство скоростей вращения спинов частиц с равными эффективными Лоренц-факторами вокруг их собственных осей стабильного спина. При этом известно, что ось стабильного спина частицы, участвующей в бетатронном движении, меняет свою ориентацию в пространстве от оборота к обороту. Последний факт послужил мотивацией проведения исследования влияния нарушения спин-прецессии на конечную статистику.

Была обнаружена систематическая структура оценки ошибки измерения, что свидетельствует о несоответствии используемой при фитировании данных модели функции-генератору данных. Также было сделано связанное с предыдущим наблюдение, что оценка частоты колебаний вертикальной компоненты поляризации пучка зависит от его вертикального сдвига по отношению к референсной орбите. Было выдвинуто две гипотезы о причине данного наблюдения:
\begin{inparaenum}[\itshape i\upshape)]
\item систематичность ошибки есть следствие различия скоростей прецессии частиц в пучке, и не связано с различием направлений их осей прецессий;
\item структура ошибки есть прямое следствие прецессии осей стабильного спина частиц в пучке.
\end{inparaenum}

Первая гипотеза основана на том факте, что, исходя из решения уравнения Т-БМТ, частота колебаний компоненты спина зависит только от величины вектора частоты, но не его направления; от последнего зависит только амплитуда колебаний. Соответственно, при выравнивании величины спин тюна при помощи секступольных полей, этот эффект должен спадать.

Вторая гипотеза основана на том факте, что статистическая оценка частоты не тождествена самой частоте колебаний компоненты, и подвержена влиянию модуляции амплитуды колебаний, вызываемой прецессией оси стабильного спина. К тому же, вариация компонент оси стабильного спина по частицам в анализируемой симуляции на много (8) порядков превосходит вариацию спин тюна, и потому вторая гипотеза является более вероятным объяснением наблюдаемого эффекта.

В процессе анализа данных было выяснено следующее:
\begin{itemize}
\item 
\item
\end{itemize}

\chapter{Оценка достоверности и достаточности данных исследования}

\chapter{Заключения и выводы}

%\bibliography{/Users/alexaksentyev/REPOS/EDM/Reports/PhD/PhDRefs}
%\bibliographystyle{vancouver}

\end{document}
